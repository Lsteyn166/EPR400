%%
%%  Department of Electrical, Electronic and Computer Engineering.
%%  EPR400/2 Project Proposal - Section 1.
%%  Copyright (C) 2011-2017 University of Pretoria.
%%

%\textbf{Motivation.}
%\textbf{Context.}
%\textbf{Technical challenge.}
%\textbf{Limitations.}
%\nocite{Haykin:Communication_Systems}
\section{Literature study}
\label{section:Lit}
In this section of the report, a brief overview of the existing research that is relevant to this project will be given, as well as a short discussion on how this existing research will be used to aid in the design of a five legged holonomic robot during the course of this project.\\

Legged vehicles are preferred to wheeled vehicles for exploration and rescue in remote areas because of their ability to cross rough terrain quickly with a severely reduced risk of getting stuck. The price to pay for this superior drive-train is far more complex electronics and movement algorithms \cite{Hidayat:Autonomous}. Instead of just driving the wheel motors and steering, each limb actuator of each leg has to be controlled to move to a calculated position.\\ 

Inverse kinematics is the mathematical process used to calculate the required angles of the limbs of a structure to reach a specific set of coordinates. This is used extensively in this project because of its ability to easily transform desired Cartesian coordinates into the required actuator angles. In the paper \cite{Oh:Analytic}, a method is discussed for solving the inverse kinematic equations involved in a seven degrees of freedom robotic arm. This method takes into account specified minimum and maximum values for each actuator and degree of freedom in order to avoid self-collision. This method can be applied to any system with a specified degrees of freedom and can therefore be simplified to solve the inverse kinematic equations for the system designed in this project.\\

To obtain the Cartesian coordinates desired at any given time, the method described in \cite{Hidayat:Autonomous}, namely Sine pattern methods is used. In \cite{Hidayat:Autonomous}, the method was applied to a quadruped robot. In the case of this project, the method is expanded to make it suitable for use on a robot with five legs. Adaptation is required because the method relies on the symmetry of a quadruped robot to schedule the lifting of the legs. Since the same symmetry does not exist in a robot with five legs, a different scheduling technique is investigated. This method will rely on information from the current position of the legs, the limits of leg movement, as well as the current horizontal tilt of the robot.\\

In the conference paper, \cite{Jain:Odometry} motion planning of omnidirectional robots is discussed and a sophisticated yet simple and efficient method of route planning is proposed. This method is based on vehicles that use three omnidirectional wheels in combination to form a resultant force vector in the desired direction. This type of vehicles differ largely in terms of locomotion when compared to a holonomic legged design used in this project, but there are a few key similarities. Both these designs can move holonomically, therefore has the ability to move in a straight line while rotating, move in an arc without rotating, or any combination of the two. These similarities mean that some of the methods discussed and applied in \cite{Jain:Odometry} can be used to aid in the design of the algorithm used in this project.\\

A team from Instituto Tecnologico de la Laguna in Mexico designed a hexapod robot \cite{Ollervides:Navigation} similar to the robot designed in this project. The purpose of the hexapod was to investigate its potential for use in exploration of areas that are hard to reach by any commonly used means of transportation. Legged vehicles are more suited to cross rough terrain, but this rough terrain complicates the design of the drive-train. In application where the surface is smooth or close to smooth, open loop control can be applied where the leg is simply moved to the desired position and it is assumed by the designer that the robot foot is making contact with the ground at this point, and therefore supporting part of the distributed robot weight. When the robot is crossing rough terrain where the surface consists of mainly bumps and holes, this assumption could be false. In such a scenario, the robot could lift another leg while under the impression that its weight is being supported by the other legs. If this is not the case, the robot could fall over and possibly damage itself or be unable to rectify itself. To avoid this problem, closed loop control should be used in the height positioning of the legs. This involves having a sensor in the system that could provide feedback on the state of the foot. The hexapod in \cite{Ollervides:Navigation} used miniature resistive force sensors attached to the bottom of each foot of the robot. The robot therefore has the ability to take analogue measurements from these sensors and determine the weight distribution of the individual feet of the robot. This data is used to confirm that all robot feet are making contact with the surface and correct the situation if this is not the case. This type of feedback is also useful when the robot is operated on slanted surfaces because of the effect that the center of gravity has on a slanted surface. A feedback sensor will be included in the design of the five legged holonomic robot in this project.\\

If the robot is operating on a slanted surface without it being aware of this and the center of gravity shifts over the lowest foot making contact, the robot could fall over even with all of its legs making contact with the surface. In a paper on reactive robot navigation \cite{Arkin:Reactive}, it is proposed that the use of a digital inclinometer can aid in solving this problem. The sensor provides information on the current tilt of the robot in two dimensions. This sensor in combination with the feedback sensors on the robot feet can be used to ensure that the robot levels itself automatically to avoid tipping over. The data collected from this sensor can also be used to collect information on the terrain. In the journal article \cite{Arkin:Reactive}, this data is used for hill climbing as well as finding valleys in unexplored areas. In order to enable the robot designed in this project to walk on slanted surfaces, a digital inclinometer will be used. Some of the reactive navigation techniques discussed in \cite{Arkin:Reactive} will also be implemented to aid the robot in navigating on slanted planes.\\

In a paper on the effects of slippery surfaces on biped robots \cite{Hyeon:Reflex}, methods on avoiding falling over of a biped robot is investigated. Since these robots have to balance themselves to stay upright, an unforeseen slippery patch on a surface could be fatal for the robot. If it were possible to foresee a slippery surface, slowing down the walking gait and increasing foot surface would help increase the traction of the robot, and therefore lower the risk of slipping. Since a five legged robot is inherently stable and there is no balancing required, slippery surfaces influences the traction of the robot but there is very low risk of falling over on level surfaces that are slippery. It is therefore suitable to just slow down movements on slippery surfaces and increasing the traction where possible.

\newpage
%% End of File.