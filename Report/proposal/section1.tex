%%
%%  Department of Electrical, Electronic and Computer Engineering.
%%  EPR400/2 Project Proposal - Section 1.
%%  Copyright (C) 2011-2017 University of Pretoria.
%%

\section{Problem statement}

\textbf{Motivation.}
The steering systems used on conventional passenger vehicles all suffer from a phenomenon called the parallel parking problem \cite{parallel:1}. This restricts the vehicle's movements to forwards and backwards while slightly turning the front wheels. The vehicle is therefore capable of moving in arcs but never sideways. The alternative to this type of movement is holonomic movement, this allows the vehicle to move in any direction from a stationary position - even sideways.  The motivation for this project is using a five-legged approach to solve the parallel parking problem while still being able to move over the bumps and small obstacles that a normal vehicle would be able to cross.

\textbf{Context.}One approach to achieve holonomic movement in vehicles is to swap the cylindrical wheels found on most vehicles for spherical wheels \cite{Omni:1}. While this solution solves the parallel parking problem, it introduces  a new restriction of only functioning properly on relatively flat surfaces unless very complicated suspension mechanisms are implemented.

\textbf{Technical challenge.}

\textbf{Limitations.}


%% End of File.

