%%
%%  Department of Electrical, Electronic and Computer Engineering.
%%  EPR400/2 Project Proposal - Section 1.
%%  Copyright (C) 2011-2017 University of Pretoria.
%%

\section{Problem statement}

\textbf{Motivation.}
The steering systems used on conventional passenger vehicles all suffer from a phenomenon called the parallel parking problem \cite{parallel:1}. This restricts the vehicle  to  forward and backward movements only, while slightly turning the front wheels. The vehicle is therefore capable of moving in arcs but never sideways. The alternative to this type of movement is holonomic movement, this allows the vehicle to move in any direction from a stationary position - even sideways.  The motivation for this project is using a five-legged approach to solve the parallel parking problem while still being able to move over the bumps and small obstacles that a normal vehicle would be able to cross.

\textbf{Context.}
One approach to achieve holonomic movement in vehicles is to swop the cylindrical wheels found on most vehicles for spherical wheels \cite{Omni:1}. While this solution solves the parallel parking problem, it introduces  a new restriction of only functioning properly on relatively flat surfaces, unless very complicated suspension mechanisms are implemented. This solution is therefore not suitable for any off-road application. This project will extend on this to solve the problem of terrain without sacrificing holonomic movement.

\textbf{Technical challenge.}
The engineering challenge in this project is the development of a control system algorithm that is accurate and complex enough to calculate all necessary outputs with the minimum number of assumptions while still being efficient enough to not cause a sluggish response. The product delivered should be a working prototype that functions in real life, not just a proof of concept.

\textbf{Limitations.}
One of the technical limitations that makes this project challenging is that the algorithm should be as efficient as possible to make maximum usage of the limited capabilities of the microcontroller. This is mainly because of the many floating point operations involved in the trigonometry of the inverse kinematic calculations. A second limitation to be taken into consideration is the limited torque that the servo motors can produce. This places a definite limitation on the maximum weight of the robot for which the legs are still able to lift the body. 


%% End of File.

