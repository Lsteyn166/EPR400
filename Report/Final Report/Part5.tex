\titleformat{\section}[frame]
{\fontsize{24pt}{28.8pt}\selectfont\bfseries} {} {5pt} {}
\titleformat{\subsection}[display]
{\fontsize{18pt}{21.6pt}\selectfont\bfseries} {} {5pt} {\thesubsection\quad}
\titleformat{\subsubsection}[frame]
{\fontsize{14pt}{16.8pt}\selectfont\bfseries} {} {5pt} {\thesubsubsection\quad}
\renewcommand\thesubsection{\arabic{subsection}.}
\renewcommand\thesubsubsection{\arabic{subsection}.\arabic{subsubsection}}
\rhead{\rightmark}
%add paragraph as extra level for experiments
\setcounter{secnumdepth}{5}
\titleformat{\paragraph}[display]
{\fontsize{12pt}{14pt}\selectfont\bfseries}{}{5pt} {\theparagraph\quad}
%add subparagraph
\titleformat{\subparagraph}[display]
{\fontsize{12pt}{14pt}}{}{5pt} {\thesubparagraph\quad}
\renewcommand\thesubparagraph{\arabic{subsection}.\arabic{subsubsection}.\arabic{paragraph}\alph{subparagraph}}

\section*{Description of product}
\markright{Part 5. Technical documentation}

This report contains the technical documentation required to design a holonomic five legged robot. The robot is built using 15 servo motors, 3D printed parts, a 32 bit microcontroller and a smartphone application that functions as a user interface and remote control.
\newpage
\tableofcontents
\markright{Part 5. Technical documentation}
\newpage
\pagenumbering{arabic}
\setcounter {section}{1}


\secun{System block}

\begin{figure}[H]
\centering
\includegraphics[scale = 0.8]{pics/Soft2.pdf}
\caption{System block diagram.}
\end{figure}

\secun{Systems level description of design}

\begin{figure}[H]
\centering
\includegraphics[scale = 0.5]{Proposal/FunctionalDiagram.pdf}
\caption{System level design.}
\end{figure}

\secun{Block diagrams of modules}
The design does not consist of modules. The system block diagram can be seen above\\
\secun{Description of modules}
The design does not consist of modules. The complete description of the design can be found in part 4 of this report.\\
\secun{Description of interfacing with other modules}
The only interfacing in this project is between the smartphone and the robot. More details on this can be found in part 4 of this report\\
\secun{Complete circuit diagram}
\begin{figure}[H]
\centering
\includegraphics[scale = 1]{pics/InputProtection.pdf}
\caption{Input protection design.}
\end{figure}
\begin{figure}[H]
\centering
\includegraphics[scale = 1]{pics/LEDDriver.pdf}
\caption{LED driver design.}
\end{figure}
\begin{figure}[H]
\centering
\includegraphics[scale = 1]{pics/PowerSupply.pdf}
\caption{Power supply design.}
\end{figure}
\begin{figure}[H]
\centering
\includegraphics[scale = 1]{pics/PowerSwitch.pdf}
\caption{Power switch design.}
\end{figure}
\secun{Description of circuit}
All of the circuit diagrams displayed above is explained in detail in part 4 of this report\\

\secun{Circuit diagrams of modules}
The circuit diagrams for all modules are shown in record 6.\\

\secun{Description of circuit diagrams}
All of the circuit diagrams displayed above is explained in detail in part 4 of this report\\

\secun{Timing diagrams}
Timing diagrams is not applicable to this project.\\

\secun{VHDL code}
VHDL code is not applicable to this project.\\

\secun{PC board layout}
No PCB was designed for this project. The veroboard planning is outlined in record 13\\

\secun{Components placement on the board}
Since the central part of the robot electronics is the microcontroller used, it is placed in the centre of the Veroboard with the support electronics around it. The DC-DC converter was placed separately from the Veroboard.

\secun{Wiring diagram of the product mounted in the enclosure}
The product has no enclusure.\\

\secun{Mechanical design}
\begin{figure}[H]
\centering
\includegraphics[scale = 1]{pics/DrawingA.pdf}
\caption{Leg A design.}
\end{figure}
\begin{figure}[H]
\centering
\includegraphics[scale = 1]{pics/DrawingB.pdf}
\caption{Leg B design.}
\end{figure}
\begin{figure}[H]
\centering
\includegraphics[scale = 1]{pics/DrawingC.pdf}
\caption{Leg C design.}
\end{figure}
\begin{figure}[H]
\centering
\includegraphics[scale = 1]{pics/DrawingBase.pdf}
\caption{Robot base design.}
\end{figure}
\begin{figure}[H]
\centering
\includegraphics[scale = 1]{pics/DrawingSpring.pdf}
\caption{torsional spring design.}
\end{figure}

\secun{Acceptance test procedure}
Although there is a large software component to the project, the output is purely mechanical. All of the results are therefore pass/fail based on the specification.

\secun{Pictures of final product}
\begin{figure}[H]
\centering
\includegraphics[angle = 270,scale = 0.1]{pics/KRL2.jpg}
\caption{Picture of final product.}
\end{figure}

\secun{User guide}
The robot is simple to operate. Once the batteries are plugged in, it is ready for Bluetooth connection.\\
\secun{List of components cost and suppliers}
This was not collected during the course of the project.\\

\secun{Description of interfacing with other entities}
There are no other entities to be interfaced with.\\

\secun{Flow diagram of software}
\begin{figure}[H]
\centering
\includegraphics[scale =1]{pics/Soft1.pdf}
\caption{Flow diagram of software.}
\end{figure}

\secun{UML diagrams}
This is not applicable to this project.\\

\secun{Complete source code}
\begin{lstlisting}[language = c]
/**
  ******************************************************************************
  * File Name          : main.c
  * Description        : Main program body
  ******************************************************************************
  ** This notice applies to any and all portions of this file
  * that are not between comment pairs USER CODE BEGIN and
  * USER CODE END. Other portions of this file, whether 
  * inserted by the user or by software development tools
  * are owned by their respective copyright owners.
  *
  * COPYRIGHT(c) 2017 STMicroelectronics
  *
  * Redistribution and use in source and binary forms, with or without modification,
  * are permitted provided that the following conditions are met:
  *   1. Redistributions of source code must retain the above copyright notice,
  *      this list of conditions and the following disclaimer.
  *   2. Redistributions in binary form must reproduce the above copyright notice,
  *      this list of conditions and the following disclaimer in the documentation
  *      and/or other materials provided with the distribution.
  *   3. Neither the name of STMicroelectronics nor the names of its contributors
  *      may be used to endorse or promote products derived from this software
  *      without specific prior written permission.
  *
  * THIS SOFTWARE IS PROVIDED BY THE COPYRIGHT HOLDERS AND CONTRIBUTORS "AS IS"
  * AND ANY EXPRESS OR IMPLIED WARRANTIES, INCLUDING, BUT NOT LIMITED TO, THE
  * IMPLIED WARRANTIES OF MERCHANTABILITY AND FITNESS FOR A PARTICULAR PURPOSE ARE
  * DISCLAIMED. IN NO EVENT SHALL THE COPYRIGHT HOLDER OR CONTRIBUTORS BE LIABLE
  * FOR ANY DIRECT, INDIRECT, INCIDENTAL, SPECIAL, EXEMPLARY, OR CONSEQUENTIAL
  * DAMAGES (INCLUDING, BUT NOT LIMITED TO, PROCUREMENT OF SUBSTITUTE GOODS OR
  * SERVICES; LOSS OF USE, DATA, OR PROFITS; OR BUSINESS INTERRUPTION) HOWEVER
  * CAUSED AND ON ANY THEORY OF LIABILITY, WHETHER IN CONTRACT, STRICT LIABILITY,
  * OR TORT (INCLUDING NEGLIGENCE OR OTHERWISE) ARISING IN ANY WAY OUT OF THE USE
  * OF THIS SOFTWARE, EVEN IF ADVISED OF THE POSSIBILITY OF SUCH DAMAGE.
  *
  ******************************************************************************
  */
/* Includes ------------------------------------------------------------------*/
#include "main.h"
#include "stm32f7xx_hal.h"

/* USER CODE BEGIN Includes */
#include <math.h>
#include <stdbool.h>

/* USER CODE END Includes */

/* Private variables ---------------------------------------------------------*/

I2C_HandleTypeDef hi2c1;

TIM_HandleTypeDef htim6;
TIM_HandleTypeDef htim7;

UART_HandleTypeDef huart1;
UART_HandleTypeDef huart3;

/* USER CODE BEGIN PV */
/* Private variables ---------------------------------------------------------*/
unsigned char RXData[15];																//Array for Bluetooth reception
unsigned char RXDataBuffer[15];													//Temp buffer for above
unsigned char X_Char;																		//Stores received X as Char
unsigned char Y_Char;																		//Stores received Y as Char
unsigned char R_Char;																		//Stores received R as Char
int X_Vect =0;																					//Stores received X as Int
int Y_Vect =0;																					//Stores received Y as Int
int R_Vect =0;																					//Stores received R as Int
int legAngles[3][15];																		//Stores the desired angle for each limb
int legAngles_id[3][15] = {{1, 2, 3, 4, 5,							//Stores the ID of each limb for sorting
											6, 7, 8, 9, 10,
											11, 12, 13, 14, 15},
											{1, 2, 3, 4, 5,										
											6, 7, 8, 9, 10,
											11, 12, 13, 14, 15},
											{1, 2, 3, 4, 5,										
											6, 7, 8, 9, 10,
											11, 12, 13, 14, 15}};

int servoOffset[] = {300,315,300,305,295,								//Array for storing the offset of each servo
											430,460,450,440,442,
											-110,-112,-105,-78,-98};		
int servoMultiplier[] = {	-360,-380,-363,-365,-370,			//Array for storing the multiplier of each servo
													-400,-400,-400,-400,-400,
													480,545,490,460,500};		
int servoCount = 0;																			//Counter used for keeping track of servo interrupts
char servoCountChar[2];
int BTCount = 0;																				//Counter for BlueTooth LED
int newPeriod = 0;

static TIM_HandleTypeDef ServoTimer;
static TIM_HandleTypeDef ServoTimer2;

unsigned char BTRewrite[] = {'X','Y','R','\r'};					//String used for debugging

struct servos {
	int theta;
	int phi;
	int alpha;
};

struct coordinates{
	double x;
	double y;
	double z;
};

struct vectors{
	int X;
	int Y;
	int R;
	bool valid;
};

double rad2deg = 57.29577951308232;
double A_length = 50;
double B_length = 106;
double C_length = 130;
double z_body = 70;
double robotRadius = 90;   

bool resetStatus[5];																		//Array used to store reset status of each leg

double X_Vector = 0;																		//Global movement vectors
double Y_Vector = 0;
double R_Vector = 0;	

double destination[2][5];																//2D array that stores X([0]) and Y([1]) values 
																												//for the destination of each foot
double currentPosition[2][5];														//2D array that stores X and Y values for the 
																												//current position of each foot
double translateLeg[2][5];															//Stores coordinates of joint 1 for each leg.
																												//Replaces TranslateLeg()
bool legAngleFlag = false;															//True when busy writing to legAngles
int bufferLevel;																				//Used to select between levels in buffer
int time = 0;																						//Keeps timing of robot movement for speed calculations
bool BTon = false;																			//Used in program to determine if Bluetooth is connected or not.
//Peripheral pin & port difinitions
#define BTPort 					GPIOE
#define BTLEDPin 				GPIO_PIN_0
#define redLEDPort 			GPIOB
#define redLEDPin 			GPIO_PIN_9
#define greenLEDPort 		GPIOB
#define greenLEDPin 		GPIO_PIN_8
#define servoPowerPort	GPIOC
#define servoPowerPin		GPIO_PIN_9

//Servo pin & port definitions
#define servo1Pin 		GPIO_PIN_12
#define servo1Port 		GPIOB
#define servo2Pin 		GPIO_PIN_13
#define servo2Port 		GPIOB
#define servo3Pin 		GPIO_PIN_14
#define servo3Port 		GPIOB
#define servo4Pin 		GPIO_PIN_15
#define servo4Port 		GPIOB
#define servo5Pin 		GPIO_PIN_8
#define servo5Port 		GPIOD
#define servo6Pin 		GPIO_PIN_9
#define servo6Port 		GPIOD
#define servo7Pin 		GPIO_PIN_10
#define servo7Port 		GPIOD
#define servo8Pin 		GPIO_PIN_11
#define servo8Port 		GPIOD
#define servo9Pin 		GPIO_PIN_12
#define servo9Port 		GPIOD
#define servo10Pin 		GPIO_PIN_13
#define servo10Port 	GPIOD
#define servo11Pin 		GPIO_PIN_14
#define servo11Port 	GPIOD
#define servo12Pin 		GPIO_PIN_15
#define servo12Port 	GPIOD
#define servo13Pin 		GPIO_PIN_6
#define servo13Port 	GPIOC
#define servo14Pin 		GPIO_PIN_7
#define servo14Port 	GPIOC
#define servo15Pin 		GPIO_PIN_8
#define servo15Port 	GPIOC

uint16_t servoPinArray[] ={	servo1Pin, servo2Pin, servo3Pin,
														servo4Pin, servo5Pin, servo6Pin,
														servo7Pin, servo8Pin, servo9Pin,
														servo10Pin, servo11Pin, servo12Pin,
														servo13Pin, servo14Pin, servo15Pin};

GPIO_TypeDef * servoPortArray[] = {	servo1Port, servo2Port, servo3Port,
																		servo4Port, servo5Port, servo6Port,
																		servo7Port, servo8Port, servo9Port,
																		servo10Port, servo11Port, servo12Port,
																		servo13Port, servo14Port, servo15Port};
/* USER CODE END PV */

/* Private function prototypes -----------------------------------------------*/
void SystemClock_Config(void);
static void MX_GPIO_Init(void);
static void MX_USART3_UART_Init(void);
static void MX_USART1_UART_Init(void);
static void MX_TIM6_Init(void);
static void MX_TIM7_Init(void);
static void MX_I2C1_Init(void);

/* USER CODE BEGIN PFP */
/* Private function prototypes -----------------------------------------------*/
void Debug(char *Array, int count);														//Sends something to UART1 for debugging
void HAL_UART_ErrorCallback(UART_HandleTypeDef *huart);				//Sends error message with Debug()
void SetupTimers(void);																				//Configure timers 6 and 7 for use in servo control
void Sort(void);																							//Sorts timer values into ascending order
void SetServo(int servo, int degrees);												//Sets the timer value for a servo from a degree value
struct servos IK( struct coordinates input);									//Inverse Kinematics function
bool CheckBounds(struct servos leg);													//Check if legs need to be reset
struct coordinates Rotate(double x, double y, double angle);	//Rotate coordinates about an angle
void TranslateLeg(void);																			//Generates constants at beginning of program
struct coordinates Vector(double x, double y, int leg, bool offset, double r);
void StartPosition(void);																			//Resets robot to default position
void HAL_UART_RxCpltCallback(UART_HandleTypeDef *huart);			//ISR for when bluetooth communication is received.
void BTDecrypt(void);																//Resturns vectors sent from android
void ServoUpdate(void);																				//Moves entire buffer over at once to servos
void Delay(int a);																						//Delays program for a cycles of 20ms
void ResetLeg(int leg);																				//Moves the specified leg back to its origin
void DestinationUpdate(int leg);															//Calculates the new destination from the input vectors and the current values

/* USER CODE END PFP */

/* USER CODE BEGIN 0 */

/* USER CODE END 0 */

int main(void)
{

  /* USER CODE BEGIN 1 */

  /* USER CODE END 1 */

  /* MCU Configuration----------------------------------------------------------*/

  /* Reset of all peripherals, Initializes the Flash interface and the Systick. */
  HAL_Init();

  /* USER CODE BEGIN Init */

  /* USER CODE END Init */

  /* Configure the system clock */
  SystemClock_Config();

  /* USER CODE BEGIN SysInit */

  /* USER CODE END SysInit */

  /* Initialize all configured peripherals */
  MX_GPIO_Init();
  MX_USART3_UART_Init();
  MX_USART1_UART_Init();
  MX_TIM6_Init();
  MX_TIM7_Init();
  MX_I2C1_Init();

  /* USER CODE BEGIN 2 */
	SetupTimers();
	StartPosition();

	TranslateLeg();

	Debug("Init done",9);

  /* USER CODE END 2 */

  /* Infinite loop */
  /* USER CODE BEGIN WHILE */
	Debug("Entering main loop",18);
	HAL_UART_Receive_IT(&huart3,RXDataBuffer,15);											//Receive commands via bluetooth in interrupt mode
  while (1)
  {
  /* USER CODE END WHILE */

  /* USER CODE BEGIN 3 */
		//Turn off all LEDs except red
		HAL_GPIO_WritePin(BTPort,BTLEDPin,GPIO_PIN_SET);
		HAL_GPIO_WritePin(greenLEDPort,greenLEDPin,GPIO_PIN_SET);
		HAL_GPIO_WritePin(redLEDPort,redLEDPin,GPIO_PIN_RESET);
		
		Debug("Waiting for Bluetooth...",24);
		
		while (BTon == false)
		{
			//Wait here for connection
		}
		//Turn off all LEDs except blue
		HAL_GPIO_WritePin(BTPort,BTLEDPin,GPIO_PIN_RESET);
		HAL_GPIO_WritePin(greenLEDPort,greenLEDPin,GPIO_PIN_SET);
		HAL_GPIO_WritePin(redLEDPort,redLEDPin,GPIO_PIN_SET);
		Debug("Turning on servos...",20);
		//Move servos to home position
		StartPosition();
		//Turn on power to servos
		HAL_GPIO_WritePin(servoPowerPort,servoPowerPin,GPIO_PIN_SET);
		//Delay for 1 second to get things started
		Delay(10);
		//Put legs in the 0,0,50 position one at a time
		for (int leg = 1; leg <6; leg++)
		{
			struct coordinates leg_Vector = Vector(0,0,leg,true,0);
			leg_Vector.z = 50;			//Lift legs above ground
			struct servos leg_servo = IK(leg_Vector);
			
			SetServo(leg,leg_servo.theta);
			SetServo(leg+5,leg_servo.phi);
			SetServo(leg+10,leg_servo.alpha);
			ServoUpdate();
			Delay(10);
		}
		//Slowly raise the body
		for (int height = 50 ; height >0 ; --height)
		{
			for (int leg = 1; leg <6; leg++)
			{
				struct coordinates leg_Vector = Vector(0,0,leg,true,0);
				leg_Vector.z = height;			//Lift legs above ground
				struct servos leg_servo = IK(leg_Vector);
				
				SetServo(leg,leg_servo.theta);
				SetServo(leg+5,leg_servo.phi);
				SetServo(leg+10,leg_servo.alpha);
				ServoUpdate();
			}
			Delay(2);
		}
		//Turn off all LEDs except green
		HAL_GPIO_WritePin(BTPort,BTLEDPin,GPIO_PIN_SET);
		HAL_GPIO_WritePin(greenLEDPort,greenLEDPin,GPIO_PIN_RESET);
		HAL_GPIO_WritePin(redLEDPort,redLEDPin,GPIO_PIN_SET);
		Debug("Ready...",8);
		
		//This is the main part that executes while the bluetooth is connected
		while(BTon)
		{
			//Get the vectors from the last transmitted message
			BTDecrypt();
			//Perform inverse kinematics
			for (int leg = 1; leg <6; leg++)
			{
//				struct coordinates leg_Vector = Vector(-X_Vect,-Y_Vect,leg,true);  //!!!!!!!!!This does not accumulate!!!!!!!!!!!!!!!!!!!!!!			
				DestinationUpdate(leg);
				struct coordinates leg_Vector;
				leg_Vector.x = currentPosition[0][leg-1];
				leg_Vector.y = currentPosition[1][leg-1];
				leg_Vector.z = 0;
				struct servos leg_servo = IK(leg_Vector);
				//Check if this is out ouf bounds
				if (CheckBounds(leg_servo))
				{
					resetStatus[leg-1] = true;
				}else{
					resetStatus[leg-1] = false;
				}
				SetServo(leg,leg_servo.theta);
				SetServo(leg+5,leg_servo.phi);
				SetServo(leg+10,leg_servo.alpha);
			}
			//Reset legs if necessary
			for (int leg = 1; leg < 6; leg++)
			{
				if (resetStatus[leg-1] == true)
				{
					//Turn on all LEDs (white)
					HAL_GPIO_WritePin(BTPort,BTLEDPin,GPIO_PIN_RESET);
					HAL_GPIO_WritePin(greenLEDPort,greenLEDPin,GPIO_PIN_RESET);
					HAL_GPIO_WritePin(redLEDPort,redLEDPin,GPIO_PIN_RESET);
					//Reset leg
					ResetLeg(leg);
				}
			}
			//Move the legs to where we want them
			ServoUpdate();
			//Turn off all LEDs except green
			HAL_GPIO_WritePin(BTPort,BTLEDPin,GPIO_PIN_SET);
			HAL_GPIO_WritePin(greenLEDPort,greenLEDPin,GPIO_PIN_RESET);
			HAL_GPIO_WritePin(redLEDPort,redLEDPin,GPIO_PIN_SET);
		}
		
		

  }
  /* USER CODE END 3 */

}

/** System Clock Configuration
*/
void SystemClock_Config(void)
{

  RCC_OscInitTypeDef RCC_OscInitStruct;
  RCC_ClkInitTypeDef RCC_ClkInitStruct;
  RCC_PeriphCLKInitTypeDef PeriphClkInitStruct;

    /**Configure the main internal regulator output voltage 
    */
  __HAL_RCC_PWR_CLK_ENABLE();

  __HAL_PWR_VOLTAGESCALING_CONFIG(PWR_REGULATOR_VOLTAGE_SCALE1);

    /**Initializes the CPU, AHB and APB busses clocks 
    */
  RCC_OscInitStruct.OscillatorType = RCC_OSCILLATORTYPE_HSE;
  RCC_OscInitStruct.HSEState = RCC_HSE_ON;
  RCC_OscInitStruct.PLL.PLLState = RCC_PLL_ON;
  RCC_OscInitStruct.PLL.PLLSource = RCC_PLLSOURCE_HSE;
  RCC_OscInitStruct.PLL.PLLM = 10;
  RCC_OscInitStruct.PLL.PLLN = 216;
  RCC_OscInitStruct.PLL.PLLP = RCC_PLLP_DIV2;
  RCC_OscInitStruct.PLL.PLLQ = 2;
  if (HAL_RCC_OscConfig(&RCC_OscInitStruct) != HAL_OK)
  {
    _Error_Handler(__FILE__, __LINE__);
  }

    /**Activate the Over-Drive mode 
    */
  if (HAL_PWREx_EnableOverDrive() != HAL_OK)
  {
    _Error_Handler(__FILE__, __LINE__);
  }

    /**Initializes the CPU, AHB and APB busses clocks 
    */
  RCC_ClkInitStruct.ClockType = RCC_CLOCKTYPE_HCLK|RCC_CLOCKTYPE_SYSCLK
                              |RCC_CLOCKTYPE_PCLK1|RCC_CLOCKTYPE_PCLK2;
  RCC_ClkInitStruct.SYSCLKSource = RCC_SYSCLKSOURCE_PLLCLK;
  RCC_ClkInitStruct.AHBCLKDivider = RCC_SYSCLK_DIV1;
  RCC_ClkInitStruct.APB1CLKDivider = RCC_HCLK_DIV4;
  RCC_ClkInitStruct.APB2CLKDivider = RCC_HCLK_DIV2;

  if (HAL_RCC_ClockConfig(&RCC_ClkInitStruct, FLASH_LATENCY_7) != HAL_OK)
  {
    _Error_Handler(__FILE__, __LINE__);
  }

  PeriphClkInitStruct.PeriphClockSelection = RCC_PERIPHCLK_USART1|RCC_PERIPHCLK_USART3
                              |RCC_PERIPHCLK_I2C1;
  PeriphClkInitStruct.Usart1ClockSelection = RCC_USART1CLKSOURCE_PCLK2;
  PeriphClkInitStruct.Usart3ClockSelection = RCC_USART3CLKSOURCE_PCLK1;
  PeriphClkInitStruct.I2c1ClockSelection = RCC_I2C1CLKSOURCE_PCLK1;
  if (HAL_RCCEx_PeriphCLKConfig(&PeriphClkInitStruct) != HAL_OK)
  {
    _Error_Handler(__FILE__, __LINE__);
  }

    /**Configure the Systick interrupt time 
    */
  HAL_SYSTICK_Config(HAL_RCC_GetHCLKFreq()/1000);

    /**Configure the Systick 
    */
  HAL_SYSTICK_CLKSourceConfig(SYSTICK_CLKSOURCE_HCLK);

  /* SysTick_IRQn interrupt configuration */
  HAL_NVIC_SetPriority(SysTick_IRQn, 0, 0);
}

/* I2C1 init function */
static void MX_I2C1_Init(void)
{

  hi2c1.Instance = I2C1;
  hi2c1.Init.Timing = 0x20404768;
  hi2c1.Init.OwnAddress1 = 0;
  hi2c1.Init.AddressingMode = I2C_ADDRESSINGMODE_7BIT;
  hi2c1.Init.DualAddressMode = I2C_DUALADDRESS_DISABLE;
  hi2c1.Init.OwnAddress2 = 0;
  hi2c1.Init.OwnAddress2Masks = I2C_OA2_NOMASK;
  hi2c1.Init.GeneralCallMode = I2C_GENERALCALL_DISABLE;
  hi2c1.Init.NoStretchMode = I2C_NOSTRETCH_DISABLE;
  if (HAL_I2C_Init(&hi2c1) != HAL_OK)
  {
    _Error_Handler(__FILE__, __LINE__);
  }

    /**Configure Analogue filter 
    */
  if (HAL_I2CEx_ConfigAnalogFilter(&hi2c1, I2C_ANALOGFILTER_ENABLE) != HAL_OK)
  {
    _Error_Handler(__FILE__, __LINE__);
  }

    /**Configure Digital filter 
    */
  if (HAL_I2CEx_ConfigDigitalFilter(&hi2c1, 0) != HAL_OK)
  {
    _Error_Handler(__FILE__, __LINE__);
  }

}

/* TIM6 init function */
static void MX_TIM6_Init(void)
{

  TIM_MasterConfigTypeDef sMasterConfig;

  htim6.Instance = TIM6;
  htim6.Init.Prescaler = 0;
  htim6.Init.CounterMode = TIM_COUNTERMODE_UP;
  htim6.Init.Period = 0;
  htim6.Init.AutoReloadPreload = TIM_AUTORELOAD_PRELOAD_DISABLE;
  if (HAL_TIM_Base_Init(&htim6) != HAL_OK)
  {
    _Error_Handler(__FILE__, __LINE__);
  }

  sMasterConfig.MasterOutputTrigger = TIM_TRGO_RESET;
  sMasterConfig.MasterSlaveMode = TIM_MASTERSLAVEMODE_DISABLE;
  if (HAL_TIMEx_MasterConfigSynchronization(&htim6, &sMasterConfig) != HAL_OK)
  {
    _Error_Handler(__FILE__, __LINE__);
  }

}

/* TIM7 init function */
static void MX_TIM7_Init(void)
{

  TIM_MasterConfigTypeDef sMasterConfig;

  htim7.Instance = TIM7;
  htim7.Init.Prescaler = 0;
  htim7.Init.CounterMode = TIM_COUNTERMODE_UP;
  htim7.Init.Period = 0;
  htim7.Init.AutoReloadPreload = TIM_AUTORELOAD_PRELOAD_DISABLE;
  if (HAL_TIM_Base_Init(&htim7) != HAL_OK)
  {
    _Error_Handler(__FILE__, __LINE__);
  }

  sMasterConfig.MasterOutputTrigger = TIM_TRGO_RESET;
  sMasterConfig.MasterSlaveMode = TIM_MASTERSLAVEMODE_DISABLE;
  if (HAL_TIMEx_MasterConfigSynchronization(&htim7, &sMasterConfig) != HAL_OK)
  {
    _Error_Handler(__FILE__, __LINE__);
  }

}

/* USART1 init function */
static void MX_USART1_UART_Init(void)
{

  huart1.Instance = USART1;
  huart1.Init.BaudRate = 115200;
  huart1.Init.WordLength = UART_WORDLENGTH_8B;
  huart1.Init.StopBits = UART_STOPBITS_1;
  huart1.Init.Parity = UART_PARITY_NONE;
  huart1.Init.Mode = UART_MODE_TX_RX;
  huart1.Init.HwFlowCtl = UART_HWCONTROL_NONE;
  huart1.Init.OverSampling = UART_OVERSAMPLING_16;
  huart1.Init.OneBitSampling = UART_ONE_BIT_SAMPLE_DISABLE;
  huart1.AdvancedInit.AdvFeatureInit = UART_ADVFEATURE_NO_INIT;
  if (HAL_UART_Init(&huart1) != HAL_OK)
  {
    _Error_Handler(__FILE__, __LINE__);
  }

}

/* USART3 init function */
static void MX_USART3_UART_Init(void)
{

  huart3.Instance = USART3;
  huart3.Init.BaudRate = 9600;
  huart3.Init.WordLength = UART_WORDLENGTH_8B;
  huart3.Init.StopBits = UART_STOPBITS_1;
  huart3.Init.Parity = UART_PARITY_NONE;
  huart3.Init.Mode = UART_MODE_TX_RX;
  huart3.Init.HwFlowCtl = UART_HWCONTROL_NONE;
  huart3.Init.OverSampling = UART_OVERSAMPLING_16;
  huart3.Init.OneBitSampling = UART_ONE_BIT_SAMPLE_DISABLE;
  huart3.AdvancedInit.AdvFeatureInit = UART_ADVFEATURE_NO_INIT;
  if (HAL_UART_Init(&huart3) != HAL_OK)
  {
    _Error_Handler(__FILE__, __LINE__);
  }

}

/** Configure pins as 
        * Analog 
        * Input 
        * Output
        * EVENT_OUT
        * EXTI
*/
static void MX_GPIO_Init(void)
{

  GPIO_InitTypeDef GPIO_InitStruct;

  /* GPIO Ports Clock Enable */
  __HAL_RCC_GPIOH_CLK_ENABLE();
  __HAL_RCC_GPIOC_CLK_ENABLE();
  __HAL_RCC_GPIOA_CLK_ENABLE();
  __HAL_RCC_GPIOB_CLK_ENABLE();
  __HAL_RCC_GPIOD_CLK_ENABLE();
  __HAL_RCC_GPIOE_CLK_ENABLE();

  /*Configure GPIO pin Output Level */
  HAL_GPIO_WritePin(GPIOC, GPIO_PIN_1|GPIO_PIN_6|GPIO_PIN_7|GPIO_PIN_8 
                          |GPIO_PIN_9, GPIO_PIN_RESET);

  /*Configure GPIO pin Output Level */
  HAL_GPIO_WritePin(GPIOA, GPIO_PIN_1, GPIO_PIN_RESET);

  /*Configure GPIO pin Output Level */
  HAL_GPIO_WritePin(GPIOB, GPIO_PIN_1|GPIO_PIN_12|GPIO_PIN_13|GPIO_PIN_14 
                          |GPIO_PIN_15|GPIO_PIN_8|GPIO_PIN_9, GPIO_PIN_RESET);

  /*Configure GPIO pin Output Level */
  HAL_GPIO_WritePin(GPIOD, GPIO_PIN_8|GPIO_PIN_9|GPIO_PIN_10|GPIO_PIN_11 
                          |GPIO_PIN_12|GPIO_PIN_13|GPIO_PIN_14|GPIO_PIN_15 
                          |GPIO_PIN_1, GPIO_PIN_RESET);

  /*Configure GPIO pin Output Level */
  HAL_GPIO_WritePin(GPIOE, GPIO_PIN_0|GPIO_PIN_1, GPIO_PIN_RESET);

  /*Configure GPIO pins : PC1 PC6 PC7 PC8 
                           PC9 */
  GPIO_InitStruct.Pin = GPIO_PIN_1|GPIO_PIN_6|GPIO_PIN_7|GPIO_PIN_8 
                          |GPIO_PIN_9;
  GPIO_InitStruct.Mode = GPIO_MODE_OUTPUT_PP;
  GPIO_InitStruct.Pull = GPIO_NOPULL;
  GPIO_InitStruct.Speed = GPIO_SPEED_FREQ_LOW;
  HAL_GPIO_Init(GPIOC, &GPIO_InitStruct);

  /*Configure GPIO pin : PA1 */
  GPIO_InitStruct.Pin = GPIO_PIN_1;
  GPIO_InitStruct.Mode = GPIO_MODE_OUTPUT_PP;
  GPIO_InitStruct.Pull = GPIO_NOPULL;
  GPIO_InitStruct.Speed = GPIO_SPEED_FREQ_LOW;
  HAL_GPIO_Init(GPIOA, &GPIO_InitStruct);

  /*Configure GPIO pins : PA4 PA5 PA6 PA7 */
  GPIO_InitStruct.Pin = GPIO_PIN_4|GPIO_PIN_5|GPIO_PIN_6|GPIO_PIN_7;
  GPIO_InitStruct.Mode = GPIO_MODE_INPUT;
  GPIO_InitStruct.Pull = GPIO_NOPULL;
  HAL_GPIO_Init(GPIOA, &GPIO_InitStruct);

  /*Configure GPIO pin : PC4 */
  GPIO_InitStruct.Pin = GPIO_PIN_4;
  GPIO_InitStruct.Mode = GPIO_MODE_INPUT;
  GPIO_InitStruct.Pull = GPIO_NOPULL;
  HAL_GPIO_Init(GPIOC, &GPIO_InitStruct);

  /*Configure GPIO pins : PB1 PB12 PB13 PB14 
                           PB15 PB8 PB9 */
  GPIO_InitStruct.Pin = GPIO_PIN_1|GPIO_PIN_12|GPIO_PIN_13|GPIO_PIN_14 
                          |GPIO_PIN_15|GPIO_PIN_8|GPIO_PIN_9;
  GPIO_InitStruct.Mode = GPIO_MODE_OUTPUT_PP;
  GPIO_InitStruct.Pull = GPIO_NOPULL;
  GPIO_InitStruct.Speed = GPIO_SPEED_FREQ_LOW;
  HAL_GPIO_Init(GPIOB, &GPIO_InitStruct);

  /*Configure GPIO pins : PD8 PD9 PD10 PD11 
                           PD12 PD13 PD14 PD15 
                           PD1 */
  GPIO_InitStruct.Pin = GPIO_PIN_8|GPIO_PIN_9|GPIO_PIN_10|GPIO_PIN_11 
                          |GPIO_PIN_12|GPIO_PIN_13|GPIO_PIN_14|GPIO_PIN_15 
                          |GPIO_PIN_1;
  GPIO_InitStruct.Mode = GPIO_MODE_OUTPUT_PP;
  GPIO_InitStruct.Pull = GPIO_NOPULL;
  GPIO_InitStruct.Speed = GPIO_SPEED_FREQ_LOW;
  HAL_GPIO_Init(GPIOD, &GPIO_InitStruct);

  /*Configure GPIO pins : PE0 PE1 */
  GPIO_InitStruct.Pin = GPIO_PIN_0|GPIO_PIN_1;
  GPIO_InitStruct.Mode = GPIO_MODE_OUTPUT_PP;
  GPIO_InitStruct.Pull = GPIO_NOPULL;
  GPIO_InitStruct.Speed = GPIO_SPEED_FREQ_LOW;
  HAL_GPIO_Init(GPIOE, &GPIO_InitStruct);

}

/* USER CODE BEGIN 4 */

/**
  * @brief  This function sends a string to UART1 to be received in a terminal
	* @param  *Array is a pointer to the string
						count is an int equal to the number of characters in the string
  * @retval None
  */
void Debug(char *Array, int count)
{
	uint8_t TXBuffer[count+1];
	for(int i = 0; i!=count;i++)
	{
		TXBuffer[i] = Array[i];
	}		
	
	TXBuffer[count] ='\r';
	TXBuffer[count+1] ='\n';
	HAL_UART_Transmit(&huart1,TXBuffer,count+2,10);
}

/**
  * @brief  This function is executed in the case of an error with the UART modules
	* @param  *huart is a handle of the UART module in question
  * @retval None
  */
void HAL_UART_ErrorCallback(UART_HandleTypeDef *huart)
{
	if (huart == &huart3)
	{
		Debug("Bluetooth Error",15);
	}else if (huart == &huart1)
	{
		Debug("Serial Error",12);
	}else{
		Debug("UART error",10);
	}
	
}

/**
  * @brief  This function configures the timers used in generating the servo signals
	* @param  None
  * @retval None
  */
void SetupTimers(void)
{
	//Timer 6
	//50Hz Interrupt
	ServoTimer.Instance =TIM6;
	ServoTimer.Init.Prescaler =40000;
	ServoTimer.Init.CounterMode = TIM_COUNTERMODE_UP;
	ServoTimer.Init.Period =54;
	ServoTimer.Init.RepetitionCounter = 0;
	HAL_TIM_Base_Init(&ServoTimer);
	HAL_TIM_Base_Start_IT(&ServoTimer);
	
	//Timer 7
	//Variable frequency interrupt
	htim7.Instance =TIM7;
	htim7.Init.Prescaler =1000;
	htim7.Init.CounterMode = TIM_COUNTERMODE_UP;
	htim7.Init.Period =160;
	htim7.Init.RepetitionCounter = 0;
	HAL_TIM_Base_Init(&htim7);
	HAL_TIM_Base_Start_IT(&htim7);
}
	

/** @brief Function for sorting the desired leg angles into ascending order 
						to make timer functions simpler
  * @param None
  * @retval None */
void Sort(void)
{
	for(int i = 0;i<15;i++)
	{
		for(int j = i+1;j<15;j++)
		{
			if(legAngles[0][i] > legAngles[0][j])
			{
				//swop the angles
				int a = legAngles[0][i];
				legAngles[0][i] = legAngles[0][j];
				legAngles[0][j] = a;
				//now swop the index
				a = legAngles_id[0][i];
				legAngles_id[0][i] = legAngles_id[0][j];
				legAngles_id[0][j] = a;
			}
		}
	}
}

/**
  * @brief  This function executes when a timer interrupts. It is used to generate servo signals
	* @param  *htim is a handle of the timer module that triggered the interrupt
  * @retval None
  */
void HAL_TIM_PeriodElapsedCallback(TIM_HandleTypeDef *htim)
	{
		
		if(htim == &htim7){
			if(legAngleFlag == true)
			{
				bufferLevel = 2;
			}else{
				bufferLevel = 1;
			}
			//Turn off
			++servoCount;			
			if (servoCount < 15)
			{
				newPeriod = (legAngles[bufferLevel][servoCount]- legAngles[bufferLevel][servoCount-1]);
				
				while (newPeriod == 0)
				{
					HAL_GPIO_WritePin(servoPortArray[legAngles_id[bufferLevel][servoCount-1]-1] ,servoPinArray[legAngles_id[bufferLevel][servoCount-1]-1],GPIO_PIN_RESET);
					++servoCount;
					newPeriod = (legAngles[bufferLevel][servoCount]- legAngles[bufferLevel][servoCount-1]);
				}
				//Adjust for time lost in this subroutine
				if (newPeriod>1)
				{
					--newPeriod;
				}
				if (newPeriod < 0 || newPeriod > 400)
				{
					newPeriod = 0;
				}
				TIM7->ARR = (newPeriod);
				HAL_GPIO_WritePin(servoPortArray[legAngles_id[bufferLevel][servoCount-1]-1] ,servoPinArray[legAngles_id[bufferLevel][servoCount-1]-1],GPIO_PIN_RESET);
			}else{
				//Turn off servo 15
				HAL_GPIO_WritePin(servoPortArray[legAngles_id[bufferLevel][14]-1] ,servoPinArray[legAngles_id[bufferLevel][14]-1],GPIO_PIN_RESET);
				//Stop the timer for the rest of the cycle
				HAL_TIM_Base_Stop(&htim7);
				if (servoCount != 0xF)
				{
					Debug("servoCount Error",16);
				}
				servoCount = 0;
			}
		}else if (htim == &htim6)
		{
			servoCount = 0;
			++time;
			//50Hz Interrupt
			//Turn all on
			HAL_GPIO_WritePin(servo1Port,servo1Pin,GPIO_PIN_SET);
			HAL_GPIO_WritePin(servo2Port,servo2Pin,GPIO_PIN_SET);
			HAL_GPIO_WritePin(servo3Port,servo3Pin,GPIO_PIN_SET);
			HAL_GPIO_WritePin(servo4Port,servo4Pin,GPIO_PIN_SET);
			HAL_GPIO_WritePin(servo5Port,servo5Pin,GPIO_PIN_SET);
			HAL_GPIO_WritePin(servo6Port,servo6Pin,GPIO_PIN_SET);
			HAL_GPIO_WritePin(servo7Port,servo7Pin,GPIO_PIN_SET);
			HAL_GPIO_WritePin(servo8Port,servo8Pin,GPIO_PIN_SET);
			HAL_GPIO_WritePin(servo9Port,servo9Pin,GPIO_PIN_SET);
			HAL_GPIO_WritePin(servo10Port,servo10Pin,GPIO_PIN_SET);
			HAL_GPIO_WritePin(servo11Port,servo11Pin,GPIO_PIN_SET);
			HAL_GPIO_WritePin(servo12Port,servo12Pin,GPIO_PIN_SET);
			HAL_GPIO_WritePin(servo13Port,servo13Pin,GPIO_PIN_SET);
			HAL_GPIO_WritePin(servo14Port,servo14Pin,GPIO_PIN_SET);
			HAL_GPIO_WritePin(servo15Port,servo15Pin,GPIO_PIN_SET);
			

			if(legAngleFlag == true)
			{
				bufferLevel = 2;
			}else{
				bufferLevel = 1;
			}
			//Set period for 1st servo
			newPeriod = legAngles[bufferLevel][servoCount];
			if (newPeriod < 0 || newPeriod > 400)
				{
					newPeriod = 0;
				}
			while (newPeriod == 0 && servoCount < 16)
			{
				++servoCount;
				newPeriod = legAngles[bufferLevel][servoCount];
				if (newPeriod < 0 || newPeriod > 400)
				{
					newPeriod = 0;
				}
			}
				
			
			TIM7->ARR = newPeriod;
			//Start the timer
			HAL_TIM_Base_Start(&htim7);
			
			//BT LED
			++BTCount;
			if (BTCount > 120) //20ms * 120 = 2.4s
			{
				//Assume BT is disconnected
				BTon = false;
				//HAL_GPIO_WritePin(servoPowerPort,servoPowerPin,GPIO_PIN_RESET);
				BTCount = 0;
//				HAL_GPIO_TogglePin(greenLEDPort,greenLEDPin);
//				HAL_GPIO_TogglePin(redLEDPort,redLEDPin);
			}
		}else{
			Debug("Timer Interrupt Error",21);
		}
	}

/**
  * @brief  This function calculates the required timer values for each of the servos
	* @param  servo is an int in the range 1:15 corresponding to a specific joint
						degrees is an int in the range 0:180 for the required position of the specific servo
  * @retval None
  */	
void SetServo(int servo, int degrees)
{
	//Set the servo with the correct index in the array 
	for (int i = 0; i<15; i++)
	{
		if (servo == legAngles_id[0][i])
		{
			legAngles[0][i] =servoOffset[servo-1] + (degrees*(servoMultiplier[servo-1])/180);
			return;
		}
	}
}

/**
  * @brief  This function does the inverse kinematic calculations for a specific leg
	* @param  input is a struct of type coordinates with members x, y and z
							corresponding to the required position of the foot relative to the hip.
  * @retval servos is a struct with members theta, phi and alpha corresponding with the 
							resulting angles required.
  */
struct servos IK( struct coordinates input)
{
	//make the return struct
	struct servos result;
	//Calculate theta
	double theta = atan2(input.x,input.y);
	//Calculate coordinates of joint 2
	double x = A_length * sin(theta);
	double y = A_length * cos(theta);
	//horizontal distance from joint 1 to foot
	double reach = sqrt(pow(input.x-x,2)+pow(input.y-y,2));
	//Absolute distance between joint 2 and foot
	double twoToFoot = sqrt(pow(input.x-x,2)+pow(input.y-y,2) +pow(input.z-z_body,2));
	//perform Cosine rule to get alpha
	double f =((pow(B_length ,2) + pow(C_length,2) - pow(twoToFoot ,2))/(2*B_length*C_length));
	double d = acos(f);
	double alpha =(rad2deg * d) + 0.5;		//+0.5 for rounding
	result.alpha = (int)alpha;
	//use sine rule to get phi
	double c = asin(C_length * sin(d) / twoToFoot);
	double e = atan2(reach ,z_body-input.z);
	double phi = 270 - c*rad2deg - e*rad2deg + 0.5; // +0.5 for rounding
	result.phi = (int)phi;
	//convert theta to degrees. Add 0.5 for rounding
	theta = (theta * rad2deg) + 0.5;
	result.theta = 180-(int)theta;
	
	return result;
}

/**
  * @brief  This function checks that all legs are within the boundaries specified and
							writes the results in a global array
	* @param  None
  * @retval None
  */
bool CheckBounds(struct servos leg)
{
	if ( leg.theta < 50 || leg.theta > 120		// 50 < theta < 120
		|| leg.phi < 90 || leg.phi > 170		// 90 < phi   < 170
		|| leg.alpha < 70 || leg.alpha> 130)		// 70 < alpha < 130
	{
		return true;				
	}else{
		return false;
	}
}

/**
  * @brief  This function rotates the given coordinates about the origin through a given angle 
	* @param  x,y are doubles corresponding to the coordinates
						angle is a double in degrees
  * @retval a coordinates struct with members x, y and z. z is not used in this case.
  */
struct coordinates Rotate(double x, double y, double angle)
{
	//Create output struct
	struct coordinates result;
	//convert angle to radians
	angle = angle/rad2deg;
	//calculate return values
	result.x = x * cos(angle) - y * sin(angle);
	result.y = x * sin(angle) + y * cos(angle);
	
	return result;
}

/**
  * @brief  This function calculates the normalized position of a leg from the 
							given movement vectors
	* @param	x is a double with the X vector
						y is a double with the Y vector
						leg is an int in the range 1:5
						offset is a bool indicating whether the coordinates should be
							denormalized
  * @retval a struct of type coordinates. The z member is unused
  */
struct coordinates Vector(double x, double y, int leg, bool offset, double r)
{
	//Create return struct
	struct coordinates result;
	//Make the trajectory polar with a magnitude and angle
	double trajectory = atan2(y,x);
	double magnitude = sqrt(pow(x,2)+pow(y,2));
	//Match the trajectory to the leg
	struct coordinates neutral;
	//Translation components
	if (offset)
	{
		neutral = Rotate(270,0,(leg-1)*72);
		struct coordinates offset= Rotate(robotRadius,0,(leg-1)*72);
		neutral.x = neutral.x - offset.x;
		neutral.y = neutral.y - offset.y;
	}else{
		neutral.x = 0;
		neutral.y = 0;
	}
	result.x = neutral.x + magnitude*cos(trajectory) + translateLeg[0][leg-1];
	result.y = neutral.y + magnitude*sin(trajectory) + translateLeg[1][leg-1];
	//+ Rotation
	struct coordinates temp = Rotate(result.x,result.y,-(leg-1)*72-5*r);
	result.x = temp.x - translateLeg[0][0];
	result.y = temp.y - translateLeg[1][0];
	if (offset)
	{
		//save the destination for  use in reset leg function later.
		destination[0][leg-1] = result.x;
		destination[1][leg-1] = result.y;
	}
	return result;
}

/**
  * @brief  This function is used once to calculate coefficients and store them in array for later use.
	* @param  None
  * @retval None
  */	
void TranslateLeg(void)
{
	for(int leg = 0; leg < 5; leg++)
	{
		translateLeg[0][leg] = robotRadius * cos(leg*72/rad2deg);
		translateLeg[1][leg] = robotRadius * sin(leg*72/rad2deg);
	}
}

/**
  * @brief  This function resets a leg if it is out of bounds
	* @param  leg is an int in the range 1:5
  * @retval None
  */
void ResetLeg(int leg)
{
	//Lift up the leg where it is currently
	struct coordinates lift;
	lift.x = destination[0][leg-1];		//destination array is updated automatically when the vector function is used.
	lift.y = destination[1][leg-1];
	lift.z = 30;
	struct servos leg_up = IK(lift);
	//Send this to servos and wait
	SetServo(leg,leg_up.theta);
	SetServo(leg+5,leg_up.phi);
	SetServo(leg+10,leg_up.alpha);
	ServoUpdate();
	Delay(10);
	//Now move the leg over to the new position
	struct coordinates reset = Vector(X_Vect,Y_Vect,leg,true,R_Vect);
	reset.z = 30;
	destination[0][leg-1] = reset.x;
	destination[1][leg-1] = reset.y;
	currentPosition[0][leg-1] = reset.x;
	currentPosition[1][leg-1] = reset.y;
	struct servos leg_reset = IK(reset);
	//Send this to servos and wait
	SetServo(leg,leg_reset.theta);
	SetServo(leg+5,leg_reset.phi);
	SetServo(leg+10,leg_reset.alpha);
	ServoUpdate();
	Delay(10);	
	//Now put down the leg where it is
	reset.z = 0;
	leg_reset = IK(reset);
	//Send this to the servos and wait
	SetServo(leg,leg_reset.theta);
	SetServo(leg+5,leg_reset.phi);
	SetServo(leg+10,leg_reset.alpha);
	ServoUpdate();
	Delay(10);
}

void StartPosition(void)
{
	//Set servo begin positions
	SetServo(1,90);
	SetServo(2,90);
	SetServo(3,90);
	SetServo(4,90);
	SetServo(5,90);
	SetServo(6,100);
	SetServo(7,100);
	SetServo(8,100);
	SetServo(9,100);
	SetServo(10,100);
	SetServo(11,90);
	SetServo(12,90);
	SetServo(13,90);
	SetServo(14,90);
	SetServo(15,90);
	ServoUpdate();
	Delay(10);
}
/**
  * @brief  This is an Interrupt service routine for when bluetooth communication is finished
  * @param  None
  * @retval None
  */
void HAL_UART_RxCpltCallback(UART_HandleTypeDef *huart)
{
	BTon = true;
	for (int i = 0;i<16;i++)
	{
		RXData[i] = RXDataBuffer[i];
	}
	HAL_UART_Receive_IT(&huart3,RXDataBuffer,15);											//Receive commands via bluetooth
}

void BTDecrypt(void)
{
	//Bluetooth reception decryption
		for(int i = 0;i<7;i++)
		{
			//Is this a valid message?
			if ((RXData[i] == 'X')&& (RXData[i+2] == 'Y')&&(RXData[i+4]=='R')&&(RXData[i+6]=='*'))
			{
//				//Turn Blue LED on
//				HAL_GPIO_WritePin(BTPort,BTLEDPin,GPIO_PIN_RESET);
//				HAL_GPIO_WritePin(servoPowerPort,servoPowerPin,GPIO_PIN_SET);
				BTon = true;
				BTCount = 0;
				//Extract the necessary commands
				X_Char = RXData[i+1];
				Y_Char = RXData[i+3];
				R_Char = RXData[i+5];
				//Clear the array
				for(int j = 0;j<14;j++)
				{
					RXData[j]=0;
				}
				//Calculate correct vectors
				X_Vect = X_Char - '5';
				Y_Vect = Y_Char - '5';
				R_Vect = R_Char - '5';
				//Debug
				HAL_UART_Transmit(&huart1,&X_Char,1,10);
				HAL_UART_Transmit(&huart1,&Y_Char,1,10);
				HAL_UART_Transmit(&huart1,&R_Char,1,10);
				HAL_UART_Transmit(&huart1,BTRewrite,4,10);
				}
		}
}

void ServoUpdate(void)
{
	//Set a flag while editing these registers
	legAngleFlag = true;
	//Sort the servo times in ascending order
	Sort();		
	//Copy the results into buffer level 2
		for (int n = 0; n<15 ;n++)
	{
	legAngles[1][n] = legAngles[0][n];
	legAngles_id[1][n] = legAngles_id[0][n];
	}
	//Clear busy flag
	legAngleFlag = false;
	//Copy into buffer level 3
	for (int n = 0; n<15 ;n++)
	{
	legAngles[2][n] = legAngles[1][n];
	legAngles_id[2][n] = legAngles_id[1][n];
	}
}

void Delay(int a)
{
	time = 0;
	while(time < a)
	{
		//wait here till delay value is reached.
	}
}

void DestinationUpdate(int leg)
{
	//Move destination to current
	currentPosition[0][leg-1] = destination[0][leg-1];
	currentPosition[1][leg-1] = destination[1][leg-1];
	//Vector to add to this
//	struct coordinates leg_Vector = Vector(-X_Vect,-Y_Vect,leg,true);
//	leg_Vector.z = 0;
	struct coordinates newVector = Vector(-X_Vect,-Y_Vect,leg,false,-R_Vect);
	//Destination = current + new
	destination[0][leg-1] = newVector.x + currentPosition[0][leg-1];
	destination[1][leg-1] = newVector.y + currentPosition[1][leg-1];
}
/* USER CODE END 4 */

/**
  * @brief  This function is executed in case of error occurrence.
  * @param  None
  * @retval None
  */
void _Error_Handler(char * file, int line)
{
  /* USER CODE BEGIN Error_Handler_Debug */
  /* User can add his own implementation to report the HAL error return state */
  while(1) 
  {
  }

	
  /* USER CODE END Error_Handler_Debug */ 
}

#ifdef USE_FULL_ASSERT

/**
   * @brief Reports the name of the source file and the source line number
   * where the assert_param error has occurred.
   * @param file: pointer to the source file name
   * @param line: assert_param error line source number
   * @retval None
   */
void assert_failed(uint8_t* file, uint32_t line)
{
  /* USER CODE BEGIN 6 */
  /* User can add his own implementation to report the file name and line number,
    ex: printf("Wrong parameters value: file %s on line %d\r\n", file, line) */
  /* USER CODE END 6 */

}

#endif

/**
  * @}
  */ 

/**
  * @}
*/ 

/************************ (C) COPYRIGHT STMicroelectronics *****END OF FILE****/

\end{lstlisting}

\secun{Explanation of software modules}
All of the software modules are explained in part 4 of this report.\\

\secun{Simulations}

\begin{lstlisting}[language = python]
#################################################################
#                                                               #
# Author: L Steyn                                               #
#                                                               #
# Code from Inverse Kinematics merged with UI test code         #
# Implemented in different threads to enable live plotting      #
#                                                               #
#################################################################

import sys
from PyQt4 import QtGui
from PyQt4 import QtCore
from matplotlib.backends.backend_qt4agg import FigureCanvasQTAgg as FigureCanvas
import matplotlib.pyplot as plt
import mpl_toolkits.mplot3d.art3d as art3d
import math
from matplotlib.patches import Circle
# Predefine Vectors
X_Vector = 0
Y_Vector = 0
Rot_Vector = 0

# Predefine variables
Z_body = 0.5                                            # Height of the robot body
A = 0.5                                                 # Leg segment A length
B = 1                                                   # Leg segment B length
C = 1                                                   # Leg segment C length
Radius = 1                                              # Radius of robot body
Z = 0
Time = 2                                                # This is the update frequency of the system in seconds
Speed = 0.1                                             # This is the speed of the robot (normalized)
destination = [0, 0, 0, 0, 0, 0, 0, 0, 0, 0]            # Init to zeros
current = [0, 0, 0, 0, 0, 0, 0, 0, 0, 0]                # Init to zeros
StepSize = 0
reset =[False,False,False,False,False]

# Create GUI class with all the components necessary to create and maintain the GUI
class GUI(QtGui.QDialog):

    def __init__(self, parent=None):                    # Main function of the GUI class. Inits everything
        super(GUI, self).__init__(parent)
        self.figure = plt.figure()                      # Create instance of matplotlib figure for GUI
        GUI.canvas = FigureCanvas(self.figure)         # Put this figure on a canvas object
        plt.ion()

        # Create UI widgets
        self.PlotButton = QtGui.QPushButton('Step')     # Plot button
        self.PlotButton.setFixedSize(50, 50)
        self.PlotButton.clicked.connect(self.Step)
        self.UpButton = QtGui.QPushButton('UP')         # Forward button
        self.UpButton.setFixedSize(50, 50)
        self.UpButton.clicked.connect(self.UP)
        self.DownButton = QtGui.QPushButton('DOWN')     # Reverse button
        self.DownButton.setFixedSize(50, 50)
        self.DownButton.clicked.connect(self.DOWN)
        self.LeftButton = QtGui.QPushButton('LEFT')     # Left button
        self.LeftButton.setFixedSize(50, 50)
        self.LeftButton.clicked.connect(self.LEFT)
        self.RightButton = QtGui.QPushButton('RIGHT')   # Right button
        self.RightButton.setFixedSize(50, 50)
        self.RightButton.clicked.connect(self.RIGHT)
        self.CWButton = QtGui.QPushButton('CW')         # Rotate clockwise button
        self.CWButton.setFixedSize(50, 50)
        self.CWButton.clicked.connect(self.CW)
        self.CCWButton = QtGui.QPushButton('CCW')       # Rotate counterclockwise button
        self.CCWButton.setFixedSize(50, 50)
        self.CCWButton.clicked.connect(self.CCW)
        self.D11 = QtGui.QLabel('X Vector:')            # X vector text label
        self.D11.setFixedWidth(75)
        self.D12 = QtGui.QLabel('0')                    # X vector numerical label
        self.D12.setFixedWidth(75)
        self.D21 = QtGui.QLabel('Y Vector:')            # Y vector text label
        self.D21.setFixedWidth(75)
        self.D22 = QtGui.QLabel('0')                    # Y vector numerical label
        self.D22.setFixedWidth(75)
        self.D31 = QtGui.QLabel('Rotation:')            # Rotation text label
        self.D31.setFixedWidth(75)
        self.D32 = QtGui.QLabel('0')                    # Rotation numerical label
        self.D32.setFixedWidth(75)
        self.AutoPlot = QtGui.QCheckBox('Auto Plot')    # Checkbox to enable plotting on press of a a directional button

        # Set all widgets in layouts
        GL = QtGui.QGridLayout()                        # Grid layout for buttons
        GL.addWidget(self.UpButton, 1, 2)
        GL.addWidget(self.DownButton, 3, 2)
        GL.addWidget(self.LeftButton, 2, 1)
        GL.addWidget(self.RightButton, 2, 3)
        GL.addWidget(self.CWButton, 5, 3)
        GL.addWidget(self.CCWButton, 5, 1)
        GL.addWidget(self.PlotButton, 2, 2)

        GL2 = QtGui.QGridLayout()                       # Grid layout for labels
        GL2.addWidget(self.D11, 1, 1)
        GL2.addWidget(self.D12, 1, 2)
        GL2.addWidget(self.D21, 2, 1)
        GL2.addWidget(self.D22, 2, 2)
        GL2.addWidget(self.D31, 3, 1)
        GL2.addWidget(self.D32, 3, 2)

        VL = QtGui.QVBoxLayout()                        # Vertical layout for grid layouts
        VL.addStretch()
        VL.addLayout(GL)
        VL.addWidget(self.AutoPlot)
        VL.addLayout(GL2)
        VL.addStretch()

        HL = QtGui.QHBoxLayout()                        # Horizontal layout for canvas and vertical layout
        HL.addWidget(self.canvas)
        HL.addLayout(VL)

        self.setLayout(HL)
        self.showMaximized()
        self.On_Start()                                 # Start update timer

    def Step(self):                                     # Function for updating the plot in the UI

        GUI.ax = self.figure.add_subplot(111, projection='3d')  # Create axis on plot
        GUI.ax.hold(True)
        ConfigurePlot()
        StepSize = Speed * Time
        for leg in range(1, 6):
            # Determine desired coordinates of a given leg from x,y vectors and rotation
            # if (current[2*leg-2] != destination[2*leg-2]):
            DestinationUpdate(leg)
            # current[2*leg-2] = X
            # current[2 * leg - 1] = Y
            reset[leg - 1] = CheckBound(current[2*leg-2], current[2 * leg - 1], leg)

        for leg in range(1, 6):
            #Reset the legs that need it
            if reset[leg-1] == True:
                ResetLeg(leg)
            X = current[2*leg-2]
            Y = current[2*leg-1]
            # Determine Servo angles for desired position
            Theta, Phi, Alpha = IK(X, Y, Z, leg)
            # Plot result
            Plotleg(Theta, Phi, Alpha, leg, False)
            # Find position of little orange markers
            x, y = RotateLeg(2.5, 0, leg)
            # Plot little orange markers for home position
            self.ax.plot([x, x], [y, y], [0, -0.05], color='#FFAF00')

        GUI.canvas.draw()                              # Refresh canvas
        print "done"

    def UP(self):                                       # Callback function for Button
        global Y_Vector
        Y_Vector += .1
        if Y_Vector > 0.5:
            Y_Vector = 0.5
        self.D22.setNum(Y_Vector)
        if self.AutoPlot.isChecked():
            self.Step()
        return

    def DOWN(self):                                     # Callback function for Button
        global Y_Vector
        Y_Vector -= .1
        if Y_Vector < -0.5:
            Y_Vector = -0.5
        self.D22.setNum(Y_Vector)
        if self.AutoPlot.isChecked():
            self.Step()
        return

    def LEFT(self):                                     # Callback function for Button
        global X_Vector
        X_Vector -= .1
        if X_Vector < -0.5:
            X_Vector = -0.5
        self.D12.setNum(X_Vector)
        if self.AutoPlot.isChecked():
            self.Step()
        return

    def RIGHT(self):                                    # Callback function for Button
        global X_Vector
        X_Vector += .1
        if X_Vector > 0.5:
            X_Vector = 0.5
        self.D12.setNum(X_Vector)
        if self.AutoPlot.isChecked():
            self.Step()
        return

    def CW(self):                                       # Callback function for Button
        global Rot_Vector
        Rot_Vector += 1
        self.D32.setNum(Rot_Vector)
        if self.AutoPlot.isChecked():
            self.Step()
        return

    def CCW(self):                                      # Callback function for Button
        global Rot_Vector
        Rot_Vector -= 1
        self.D32.setNum(Rot_Vector)
        if self.AutoPlot.isChecked():
            self.Step()
        return

    def On_Timer(self):
        """Executed when the timer runs out"""
        self.Step()                                     # Automatically take the next step
        print "tick"

    def On_Start(self):
        """This module is called to start the timer and multithreading services"""
        # Start and init timer
        self.timer = QtCore.QTimer()
        self.timer.timeout.connect(self.On_Timer)
        self.timer.start(Time*1000)

        # Start the necessary services
        for leg in range(1, 6):
            X, Y = Vector(-X_Vector, -Y_Vector, leg, -Rot_Vector, True)
            destination[2 * leg - 2] = X
            destination[2 * leg - 1] = Y
        # End of class GUI


def DestinationUpdate(leg):
    """This module updates the global update array"""
    # X, Y = Vector(-X_Vector, -Y_Vector, leg, -Rot_Vector,True)
    # The stepsize should now be added to the current array in the direction of movement
    current[2 * leg - 2] = destination[2 * leg - 2]
    current[2 * leg - 1] = destination[2 * leg - 1]
    global StepSize
    StepSize = math.sqrt(X_Vector**2+Y_Vector**2+Rot_Vector**2)  # Absolute stepsize from input vectors
    x, y = Vector(-X_Vector, -Y_Vector, leg, -Rot_Vector, False)
    destination[2 * leg - 2] = x + current[2 * leg - 2]
    destination[2 * leg - 1] = y + current[2 * leg - 1]

# Functions from inverse kinematics calculations

def Plotleg(Theta, Phi, Alpha, leg, Dim):
    ""
    "This function plots a leg after the inverse kinematics are calculated"
    # Call the translation function
    offsetx, offsety = TranslateLeg(leg)

    # Plot segment A
    if Dim == True:
        plotColour = '#FFAFAF'
    else:
        plotColour = '#FF0000'
    LegX = [offsetx, offsetx + A * math.sin(Theta)]
    LegY = [offsety, offsety + A * math.cos(Theta)]
    LegZ = [Z_body, Z_body]
    GUI.ax.plot(LegX, LegY, LegZ, plotColour)

    # Plot segment B
    if Dim == True:
        plotColour = '#AFFFAF'
    else:
        plotColour = '#00FF00'
    LegX = [offsetx + A * math.sin(Theta), offsetx + A * math.sin(Theta) + B * math.sin(Phi) * math.sin(Theta)]
    LegY = [offsety + A * math.cos(Theta), offsety + A * math.cos(Theta) + B * math.sin(Phi) * math.cos(Theta)]
    LegZ = [Z_body, Z_body + B * math.cos(Phi)]
    GUI.ax.plot(LegX, LegY, LegZ, plotColour)

    # Plot segment C
    if Dim == True:
        plotColour = '#AFAFFF'
    else:
        plotColour = '#0000FF'
    LegX = [offsetx + A * math.sin(Theta) + B * math.sin(Phi) * math.sin(Theta),
            offsetx + A * math.sin(Theta) + B * math.sin(Phi) * math.sin(Theta) + C * math.sin(
                Phi + Alpha - math.radians(90)) * math.sin(Theta)]
    LegY = [offsety + A * math.cos(Theta) + B * math.sin(Phi) * math.cos(Theta),
            offsety + A * math.cos(Theta) + B * math.sin(Phi) * math.cos(Theta) + C * math.sin(
                Phi + Alpha - math.radians(90)) * math.cos(Theta)]
    LegZ = [Z_body + B * math.cos(Phi), Z_body + B * math.cos(Phi) + C * math.cos(Phi + Alpha - math.radians(90))]
    GUI.ax.plot(LegX, LegY, LegZ, plotColour)
    # print (LegX[1], LegY[1], LegZ[1]), "\n\r"
    return

def IK(X, Y, Z, leg):
    ""
    "This function calculates the servo positions required for specific coordinates"
    # Call the translatiom function
    Theta = math.atan2(X, Y)  # Calculate Theta
    # print "Theta", leg, "=", math.degrees(Theta)
    ###Calculate Phi & Alpha:
    x1 = A * math.sin(Theta)
    y1 = A * math.cos(Theta)
    z1 = Z_body
    Reach = math.sqrt((X - x1) ** 2 + (Y - y1) ** 2)  # Horizontal Distance from Joint 2 to foot
    TwoToFoot = math.sqrt((X - x1) ** 2 + (Y - y1) ** 2 + (Z - z1) ** 2)  # Absolute Distance from Joint 1 to foot
    # print("TwoToFoot")
    # print(TwoToFoot)
    x = (B ** 2 + C ** 2 - TwoToFoot ** 2) / (2 * B * C)
    # print(x)
    d = math.acos(x)
    # print("d")
    # print(math.degrees(d))
    Alpha = 270 - math.degrees(d)
    Alpha = math.radians(Alpha)
    # print "Alpha", leg, "=", math.degrees(Alpha)
    c = math.asin(C * math.sin(d) / TwoToFoot)
    # print("c")
    # print(math.degrees(c))
    e = math.atan2(Reach, Z_body - Z)
    # print("e")
    # print(math.degrees(e))
    Phi = 180 - math.degrees(c) - math.degrees(e)
    Phi = math.radians(Phi)
    # print "Phi  ", leg, "=", math.degrees(Phi)
    return Theta, Phi, Alpha

def Rotate(xi, yi, angle):
    ""
    "This function rotates the coordinates of a specific point around the origin by a certain angle(degrees)"
    Angle = math.radians(angle)
    xo = xi * math.cos(Angle) - yi * math.sin(Angle)
    yo = xi * math.sin(Angle) + yi * math.cos(Angle)
    return xo, yo

def RotateLeg(x, y, leg):
    ""
    "This function rotates the coordinates of a specific leg to its required position"

    Angle = math.radians((leg - 1) * 72)
    X = x * math.cos(Angle) - y * math.sin(Angle)
    Y = x * math.sin(Angle) + y * math.cos(Angle)
    return X, Y

def TranslateLeg(leg):
    ""
    "This function is called from the plotleg function to set the necessary offsets for plotting."
    "It calculates the offset distance from shoulder joint to robot centre"
    x = Radius * math.cos(math.radians(72 * (leg - 1)))
    y = Radius * math.sin(math.radians(72 * (leg - 1)))
    return x, y

def Vector(x, y, leg, Rot, Offset):
    ""
    "This function determines the required position of the foot given a specific vector and rotation"
    # First do the translation vector
    Angle = math.atan2(y, x)
    Magnitude = math.sqrt(x ** 2 + y ** 2)  # Calculate actual magnitude
    if (Magnitude > 0.5):  # Limit magnitude to 0.5
        Magnitude = 0.5
    if Offset:
        # Find Neutral position of each leg:
        NeutralX, NeutralY = RotateLeg(2.5, 0, leg)
        # Move leg to the edge of the robot:
        a, b = RotateLeg(Radius, 0, leg)
        NeutralX = NeutralX - a
        NeutralY = NeutralY - b
    else:
        NeutralY = 0
        NeutralX = 0

    X = NeutralX + Magnitude * math.cos(Angle)
    Y = NeutralY + Magnitude * math.sin(Angle)

    # Now the rotation part
    offsetx, offsety = TranslateLeg(leg)  # Move system origin to robot center
    X = X + offsetx
    Y = Y + offsety
    X, Y = Rotate(X, Y, -10 * Rot)  # Rotate around origin
    X = X - offsetx
    Y = Y - offsety

    if Offset:
        current[2 * leg - 2] = X
        current[2 * leg - 1] = Y

    return X, Y

def CheckBound(x, y, leg):
    ""
    "This function checks if a leg needs to be reset by determining whether it is within range from its neutral position"
    NeutralX, NeutralY = RotateLeg(1.5, 0, leg)
    deltaX = NeutralX - x
    deltaY = NeutralY - y
    absDist = math.sqrt(deltaX**2+deltaY**2)
    if absDist > 0.51:
        return True
    else:
        return False

def ResetLeg(leg):
    ""
    "This function resets a leg to the postion furthest in the direction of robot movement"
    #Pick up the leg
    Z = 0.5
    X = destination[2*leg-2]
    Y = destination[2*leg-1]
    # Determine Servo angles for desired position
    Theta, Phi, Alpha = IK(X, Y, Z, leg)
    # Plot result
    Plotleg(Theta, Phi, Alpha, leg, True)
    # plt.pause(0.5)
    # GUI.canvas.draw()
    #Move to reset position
    #Angle =
    #x =
    #y = -(current[2 * leg - 1])
    X, Y = Vector(X_Vector, Y_Vector, leg, Rot_Vector, True)
    current[2*leg-2] = X
    current[2*leg-1] = Y
    destination[2 * leg - 2] = X
    destination[2 * leg - 1] = Y
    Theta, Phi, Alpha = IK(X, Y, Z, leg)
    Plotleg(Theta, Phi, Alpha, leg, True)
    # GUI.canvas.draw
    #Put leg down
    Z = 0
    Theta, Phi, Alpha = IK(X, Y, Z, leg)
    Plotleg(Theta, Phi, Alpha, leg, False)
    return

def ConfigurePlot():
    ""
    "This function creates a 3D plot and configures axes and labels"
    blank = [0, 0]
    GUI.ax.plot(blank, blank, blank, label="Segment A", color='#FF0000')
    GUI.ax.plot(blank, blank, blank, label="Segment B", color='#00FF00')
    GUI.ax.plot(blank, blank, blank, label="Segment C", color='#0000FF')
    GUI.ax.plot(blank, blank, blank, label="Chassis", color='#AF00FF')
    GUI.ax.plot(blank, blank, blank, label="Feet zone", color='#FFAF00')

    p = Circle((0, 0), Radius, color='#AF00FF')
    GUI.ax.add_patch(p)
    art3d.pathpatch_2d_to_3d(p, z=Z_body, zdir="z")

    GUI.ax.legend()
    GUI.ax.set_xlabel('X')
    GUI.ax.set_ylabel('Y')
    GUI.ax.set_zlabel('Z')
    GUI.ax.axis([-2.5, 2.5, -2.5, 2.5])
    GUI.ax.set_zlim3d(0, 5)

if __name__ == '__main__':                              # Start of main application
    app = QtGui.QApplication(sys.argv)
    main = GUI()                                        # Start a GUI instance
    main.show()

    sys.exit(app.exec_())                               # Stop when GUI is closed
\end{lstlisting}

\secun{Hardware, software and operating system requirements}
The smartphone application requires Android 6 or above to function

\secun{Software user guide}

The user does not interact with the software.\\

\secun{Software acceptance test procedure}
Although there is a large software component to the project, the output is purely mechanical. All of the results are therefore pass/fail based on the specification.

\secun{Experimental data}
All of the experimental data can be found in part 4 of this report.
