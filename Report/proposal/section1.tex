%%
%%  Department of Electrical, Electronic and Computer Engineering.
%%  EPR400/2 Project Proposal - Section 1.
%%  Copyright (C) 2011-2017 University of Pretoria.
%%

\section{Problem statement}

\textbf{Motivation.}
Robots used for exploration need to be highly manoeuvrable to cross terrain not possible for humans and cars. The motivation for this project is to develop a robot using five legs, capable of crossing rough terrain, that can move holonomically. A robot like this can be used to explore autonomously and can also be used to carry supplies or equipment to remote places.

\textbf{Context.}
A vehicle is considered holonomic if it does not suffer from a phenomenon called the parallel parking problem \cite{parallel:1}. This phenomenon restricts the vehicle to  forward and backward movements only, while slightly turning the front wheels. The vehicle is therefore capable of moving in arcs but never sideways. One approach to achieve holonomic movement in vehicles is to swop the cylindrical wheels found on most vehicles for spherical wheels \cite{Omni:1}. While this solution solves the parallel parking problem, it introduces  a new restriction of only functioning properly on relatively flat surfaces, unless very complicated suspension mechanisms are implemented. This solution is therefore not suitable for any off-road application. This project will extend on this to solve the problem of terrain without sacrificing holonomic movement. LS3 \cite{LS3} and BigDog  \cite{BigDog} are examples of existing legged robots that function in a way that mimics the way four legged animals walk. These robots were both developed by Boston Dynamics to help humans carry loads across terrain that is not accessible by car. They can therefore follow someone on a foot trail as well as being able to navigate to a given location using GPS.

\textbf{Technical challenge.}
The aim of this project is to build a five legged holonomic robot which is capable of moving in any direction from a standing position as well as being able to rotate about its own axis. The control system of the robot will consist of a processor calculating all the required joint angles of all the legs to move the legs in a way that the robot moves in the desired direction. The engineering challenge in this project is the development of a control system algorithm that is fast enough to make the robot react in real time while still being thorough enough  to ensure that the robot does not damage itself or its surroundings while moving.

\textbf{Limitations.}
One of the technical limitations that makes this project challenging is the trade-off considering the length of the robot legs. Longer legs will mean more manoeuvrability for the robot and faster movements but will also mean increased torque required by the servo motors. This causes a larger power consumption and therefore a shorter battery life. It is also easier to construct a larger robot but this will again eventually lead to bigger power requirements.


%% End of File.

