\subsubsection{Summary of the work}
The mathematical analysis of the generic robot body was done in such a way that the model could be changed parametrically to allow for greater flexibility in the design process. This was applied and developed in a Python simulation which eventually ended up to include a user interface and a plot of the robot body in action. Once the walking algorithm was implemented and optimized, the work was implemented in C to be used in the microcontroller that would control the robot. Algorithms were designed and implemented to control all 15 of the servo motors with as little hardware and software resources required as possible. An android application was developed to be a user interface and act as a remote control for the robot. Instructions are sent to the robot using Bluetooth, which are received by a Bluetooth module connected to the microcontroller with a serial connection. Legs and a body was 3D-printed to fit the servo motors and the robot was able to move its first legs. The movement of legged vehicles on slippery surfaces was investigated and the results were surprising. The investigation was somewhat limited by the available torque from the servo motors.\\
\subsubsection{Summary of the observations and findings}
The robot is able to walk but not in the way originally intended. Although holonomic motion is possible and those design goals have been met, the robot struggles to move properly. This is a result of the robot being much heavier than anticipated. The extra strain on the servo motors causes a high power demand from the servo motors which the DC-DC converter is not always able to supply. The result in the rail voltage is that dips occur when the robot requires strength for more than one leg at a time. The dip in voltage reduces the servo torque, making the situation even worse. A partial solution to this is to keep the robot as light as possible and severely limit the speed of all movements. This somewhat helps but the robot is no longer able to meet the mission critical specification for speed. The movement accuracy also suffers from the same problem.\\
\subsubsection{Suggestions for future work}
The work could be continued and largely improved by getting the robot to walk properly without exchanging the servo motors with stronger ones. This could possibly be done by changing the gear ratio on the outer two limbs to sacrifice some range for torque. This will allow the design to keep the low power consumption it was aiming for while being able to work as intended in the design.