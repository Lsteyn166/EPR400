\newpage
\subsubsection{Interpretation of results}
The results recorded in the previous section of this report indicate that the robot is able to function as a holonomic robot, therefore it is able to move in any direction without first rotating and rotate without translating. The results that measure the performance of the robot, however, show that it does not do it nearly as good as it was designed to do it. The maximum speed of the robot, for example, is far below what was aimed for originally. The fact that the robot is able to correctly execute commands shows that the theoretical design and algorithm is valid. In the case of the speed requirement, it is the mechanical design that was lacking. The robot was far heavier that planned for and the servo motors used struggle to carry the load. The only way to make the robot function is to slow all of the movements down. The test on slippery surfaces had unexpected results as well. The expectation was that the robot might struggle to walk because the legs would slip on the floor and it would slip when propelling itself forward. Instead the low friction causes the legs to want to spread outwards, causing more strain on the servo motors. The choice of servo motor implemented in the robot caused the robot to not work as good as it should have. If more powerful servo motors were implemented, the power consumption would have been significantly higher but torque would have been sufficient.  

\subsubsection{Aspects to be improved}
The mechanical drive system for the legs could be improved. Torque at the critical joint is not sufficient and the full range of motion is not used. A solution that would not require a complete redesign would be to change the gear ratio at the critical joint to have a more powerful actuator with a smaller range of motion

\subsubsection{Strong points}

\subsubsection{Under which circumstances will the current system fail?}

\subsubsection{Design ergonomics}

\subsubsection{Health and safety aspects of the design}

\subsubsection{Social and legal impact and benefits of the design}

\subsubsection{Environmental impact and benefits of the design}