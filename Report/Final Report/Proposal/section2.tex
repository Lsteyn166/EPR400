%%
%%  Department of Electrical, Electronic and Computer Engineering.
%%  EPR400/2 Project Proposal - Section 2.
%%  Copyright (C) 2011-2017 University of Pretoria.
%%

\section*{2. Project requirements}
\addcontentsline{toc}{subsection}{2. Project requirements}

\secunderline{ELO 3: Design part of the project}

\subsection*{Mission requirements of the product}A five legged holonomic robot will be built in this project. The requirements of the robot that would determine whether the project is successful can be summarized in the list below.
\begin{itemize}
\item The robot should be able to move in any direction from a stationary position.
\item The robot should be able to rotate about its own axis while remaining in the same position.
\item The robot should be controlled remotely by using a smartphone application.
\item The robot should use five legs to execute any of the required movements.
\item The robot should be able to move on both smooth and coarse surfaces.
\item The robot should be able to move on both flat and slanted surfaces.
\end{itemize}

\subsection*{Student tasks: design}
The tasks that are vital to ensuring that the product meets the mission critical requirements are listed below.
\begin{itemize}
\item The mathematical analysis and design for the movement of the legs should be done on paper.
\item The design should then be implemented in a graphical mathematics package such as Python for further refinement.
\item Once the algorithm design is sound, electronic design can commence in a simulation environment such as LTSpice.
\item The design can then be implemented in electronics using a microcontroller and support electronics together with some driving circuitry.
\item A smartphone application should be developed to remotely control the robot.
\item Each subsystem should be tested for isolated functionality as well as interaction with other subsystems to make sure all requirements are met. 
\end{itemize}

\vspace{5em}
\secunderline{ELO 4: Investigative part of the project}

\subsection*{Research questions}
The investigative section on this project will focus on the amount of legs of a legged robot and the effect on stability. Do five legs provide more stability when walking on slippery surfaces? Can the robot still work without some of its legs?

\subsection*{Student tasks: experimental work}
The experiments that will be conducted to address the research questions above are listed below.\\

An experimental setup to test the ability of the robot to move on a variety of surfaces is required.
\begin{itemize}
\item The robot will be placed on both smooth and coarse flat surfaces to perform the same manoeuvres.
\item The robot should be able to perform these manoeuvres with a similar degree of difficulty and in similar time.
\end{itemize}
Another experimental setup is required to test the robot's ability to handle small obstacles and bumps.
\begin{itemize}
\item The robot will be placed on a surface with bumps and small obstacles such as rocks.
\item It should be able to execute the same set of manoeuvres mentioned in the experimental setup above in similar time.
\end{itemize}
\newpage 

%% End of File.


