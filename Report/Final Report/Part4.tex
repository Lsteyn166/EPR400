\newpage
\setcounter{subsection}{0}
%Change formatting for tis section

\titleformat{\section}[frame]
{\fontsize{24pt}{28.8pt}\selectfont\bfseries} {} {5pt} {}
\titleformat{\subsection}[display]
{\fontsize{18pt}{21.6pt}\selectfont\bfseries} {} {5pt} {\thesubsection\quad}
\titleformat{\subsubsection}[display]
{\fontsize{14pt}{16.8pt}\selectfont\bfseries} {} {5pt} {\thesubsubsection\quad}
\renewcommand\thesubsection{\arabic{subsection}.}
\renewcommand\thesubsubsection{\arabic{subsection}.\arabic{subsubsection}}
\rhead{\rightmark}
%add paragraph as extra level for experiments
\setcounter{secnumdepth}{5}
\titleformat{\paragraph}[display]
{\fontsize{12pt}{14pt}\selectfont\bfseries}{}{5pt} {\theparagraph\quad}
%add subparagraph
\titleformat{\subparagraph}[display]
{\fontsize{12pt}{14pt}}{}{5pt} {\thesubparagraph\quad}
\renewcommand\thesubparagraph{\arabic{subsection}.\arabic{subsubsection}.\arabic{paragraph}\alph{subparagraph}}


%Start
\section*{Part 4. Main report}
\addcontentsline{toc}{section}{Part 4. Main report}
\markright{Part 4. Main report}
\secun{Literature study}
\pagenumbering{arabic}
\captionsetup[table]{position=bottom}
\input{Lit.tex}

\secun{Approach}
This section outlines the initial approach to solve the functions shown in the functional block diagram. The functional block diagram of the project can be found in Figure \ref{fig:Func} (in Part 3) in this document.

The core of the functional block diagram is the processing unit, which is implemented in the form of a 32 bit microcontroller with an algorithm embedded in firmware. The microcontroller that will be implemented in the final design will be determined by the amount of digital inputs and outputs (IO) required, as well as the amount of mathematical equations required per second.

The microcontroller will act on instructions received through a wireless communication channel that originate from a smartphone application. The application will only act as a user interface for the robot and will output basic instructions for the robot to follow.

\subsubsection{Design alternatives}
The two wireless technologies built into most modern smartphones that best suit the needs of the communication channel is Wi-Fi and Bluetooth. Journal article \cite{Dhawan:Analogy} makes some comparisons between these two similar but different technologies. Wi-Fi has a much higher bandwidth than Bluetooth as well as providing for a far superior range. Wi-Fi networks are generally much more complex than Bluetooth networks and are therefore not generally suited for low level implementation. However, Bluetooth is better suited for peer to peer communications, whereas Wi-Fi is mostly intended to be used with a router. Network security for Wi-Fi is much more advanced and therefore much more secure than that of Bluetooth.

After the commands are passed from the smartphone to the microcontroller, the microcontroller interfaces to actuators to execute the movements desired by the user. When considering possible actuators that can be used, three possible solutions come to mind. These are servo motors, geared DC motors and stepper motors. Geared DC motors are small, light and easy to implement but lack feedback on shaft position. Such feedback needs to be implemented manually to be able to make controlled movements. Stepper motors have fixed step resolution, therefore making controlled movements is easier once the current position is known. The drawback of these motors is that a homing mechanism and routine needs to be implemented to find a reference. Stepper motors are also heavy and require high current even when stationary. Servo motors are basically geared DC motors with a positional feedback control system built-in. Once the motor  receives power and a desired position, it will attempt to reach the desired position. The disadvantage is that these have a limited range of motion.

The overall shape of the robot, more specifically, the placement of legs around the robot will greatly influence the robot's performance when moving forward in a straight line, sideways or over obstacles.

A popular design in hexapod robots is placing the legs in groups of three on opposite sides. This means that the legs are all aligned in a similar direction while the body is usually a long rectangular shape, similar to that found on insects. The advantage of this design lies mainly in the ability to move forward very quickly since the legs are placed optimally for forward motion. While holonomic movement is possible, it is usually much slower than forward or backward motion. This design is usually more suited to robots with an even number of legs due to the symmetry of the design.

An alternative design approach is to space the legs evenly around the robot's chassis. The robot's chassis would normally be round, or a polygon shape, with the same number of sides as legs. In the case of a five legged robot, the legs would be positioned $\frac{360^o}{5} = 72^o $ apart.
This equally spaced design approach has the disadvantage of not being particularly fast in any given direction, but the advantage that it can manage the maximum speed in a specific direction. Figure \ref{fig:Body_layout} shows an examples of both layouts.
\begin{figure*}[t!]
    \centering
    \begin{subfigure}[t]{0.5\textwidth}
        \centering
        \includegraphics[scale = 0.6]{pics/Body_layout_hex.pdf}
        \caption{Symmetric chassis layout.}
    \end{subfigure}%
    ~ 
    \begin{subfigure}[t]{0.5\textwidth}
        \centering
        \includegraphics[scale = 0.5]{pics/Body_Layout.pdf}
        \caption{Equally spaced chassis layout.}
    \end{subfigure}
    \caption{Examples of two different chassis layout options for legged robots.}
    \label{fig:Body_layout}
\end{figure*}

\subsubsection{Preferred solution}
Since the data transfer in the communication channel between the smartphone and the robot consists only of simple commands and the required bandwidth is therefore low, Bluetooth will be used instead of Wi-Fi. Bluetooth has the advantage of being less expensive and far less complicated to implement. The added security that Wi-Fi offers is not required for this application. The disadvantage of using Bluetooth instead of Wi-Fi is the very limited range that Bluetooth offers.\\

The commands received via Bluetooth will be actuated with the use of servo motors. These compact units are ideal for this scenario as they have all of the advantages of geared DC motors with all of the required control systems built-in. Servo motors have the advantage that they are usually light for the torque they can provide, which is ideal for a robot where weight is often a problem.\\

This will all be implemented in a circular shaped chassis with equally spaced legs as this design is better suited for holonomic movements than the symmetric design as discussed above.



\secun{Design and implementation}
\subsubsection{Background}
A large part of the design of a legged robot depends on the design of a simple leg - more specifically the degrees of freedom that a single leg has. The degrees of freedom that a single leg has is determined by the amount of dimensions that the leg can make a controlled movement in, independently from any other dimensions. There are two common underlying designs in robot leg design - these are for two and three degrees of freedom respectively.\\

Figure \ref{fig:2DOF} shows an example of a common design that has two degrees of freedom. The design therefore contains two joints in one leg. In this example, the first joint is positioned vertically to form the hip op the robot leg. This allows for the side-to-side motion required for walking. The second joint is positioned horizontally to allow the lowest limb to move up and down. The two degrees of freedom controlled by this leg design is therefore the horizontal angle of the leg and the height of the foot. These two (and only these two) parameters can therefore be controlled completely independently.
\FloatBarrier
\begin{figure}[h]
\centering
\includegraphics[scale = 1]{pics/2DOF.pdf}
\caption{Example of a leg design with two degrees of freedom.}
\label{fig:2DOF}
\end{figure}
\FloatBarrier
An example of a robotic leg with three degrees of freedom can be seen in Figure \ref{fig:3DOF}. This design is similar to that of the leg with two degrees of freedom seen in Figure \ref{fig:2DOF}, with the only addition being a second horizontal joint further down from the first. This additional limb means that the horizontal distance from the foot to the hip can be controlled as well as the height of the foot. These two controlled parameters, together with the horizontal leg angle which can also be controlled independently as in the case of the two degrees of freedom design, means that this design has a total of three degrees of freedom.
\FloatBarrier
\begin{figure}[h]
\centering
\includegraphics[scale = 1]{pics/3DOF.pdf}
\caption{Example of a leg design with three degrees of freedom.}
\label{fig:3DOF}
\end{figure}
\FloatBarrier


The design with three degrees of freedom has the advantage of being able to lift a leg without altering the horizontal position of the leg. This means that the robot is able to walk without altering the height of the body of the robot. The robot is therefore able to cross rougher terrain much better because of the ability to alter the height of the robot feet as the terrain requires. With the design that has two degrees of freedom, the foot height is a function of the horizontal extension of the leg. With this design the main advantage is the simplicity - both mechanically and in software. The cost and power consumption will also be much lower because of the reduced amount of actuators.\\

Due to the much greater flexibility of the design with three degrees of freedom and the ability to cross rougher terrain, this will be the platform implemented in the final design.\\

\subsubsection{Theoretical analysis and modelling}
\label{sec:theory}
Since each leg has three degrees of freedom, there are three angles that need to be controlled in order to move the leg to a desired position. These will be referred to as theta ($\theta$), phi($\phi$) and alpha($\alpha$) throughout this document. Theta describes the horizontal angle between the tangent of the body and leg segment A as shown in Figure \ref{fig:Leg_design_2}. 

\FloatBarrier
\begin{figure}[h]
\centering
\includegraphics[scale = 1]{pics/Leg_design_2.pdf}
\caption{Top view of the leg design model for mathematical analysis.}
\label{fig:Leg_design_2}
\end{figure}
\FloatBarrier

Phi describes the angle between leg segments A and B viewed directly from the side. In the same plane, alpha describes the angle between leg segments B and C. Both of these angles can be seen in Figure \ref{fig:Leg_design}.

\FloatBarrier
\begin{figure}[h]
\centering
\includegraphics[scale = 1]{pics/Leg_design.pdf}
\caption{Side view of the leg design model for mathematical analysis.}
\label{fig:Leg_design}
\end{figure}
\FloatBarrier

In order to control the robot in a Cartesian world it is necessary to find a transfer function for a leg that can translate a coordinate vector in the domain $[\theta,\phi,\alpha]$ to a vector in the domain $[x,y,z]$. This is done through mathematical analysis of a single leg.\\

By using Figure \ref{fig:Leg_design_2} and some simple trigonometry, theta can be found to be

\begin{align}
\label{eq:IKstart}
\theta = arctan\Big(\frac{x}{y}\Big)
\end{align}

where \textit{x}, \textit{y} and \textit{z} are the desired coordinates for the foot of the relevant leg. It is important to note that the hip of the leg where section A meets the chassis is the origin of the Cartesian system for this analysis. This can be seen in Figure \ref{fig:Leg_design_3}

\FloatBarrier
\begin{figure}[h]
\centering
\includegraphics[scale = 1]{pics/Leg_design_3.pdf}
\caption{Annotated top view of the leg design model for mathematical analysis.}
\label{fig:Leg_design_3}
\end{figure}
\FloatBarrier

In order to do an analysis on phi and alpha, it is necessary to add a few construction lines to the generic model in Figure \ref{fig:Leg_design}. The additions can be seen in Figure \ref{fig:Leg_design_4} and the additions are explained below.

\begin{figure}[h]
\centering
\includegraphics[scale = 1]{pics/Leg_design_4.pdf}
\caption{Annotated side view of the leg design model for mathematical analysis.}
\label{fig:Leg_design_4}
\end{figure}

From Figure 8:
\begin{itemize}
\item $Z_{body}$ is the nominal height difference between the robot chassis and ground. This value is a constant.
\item \textit{Reach} is the horizontal distance between the hip and the foot.
\item \textit{D} is an imaginary line between joint 2 and the foot that completes the triangle \textit{BCD} that will form the basis of this analysis.
\item $E_x$ is the horizontal distance between joint 2 and the foot.
\item $E_y$ is the horizontal line that completes triangle $DE_xE_y$. Quantitatively $E_y = Z_{body}$ since the ground is assumed to be level in this analysis.
\item \textit{c} is the angle between $B$ and $D$.
\item \textit{b} is the angle between $C$ and $D$.
\item \textit{e} is the angle between $E_y$ and $D$.
\end{itemize}

\begin{align}
\therefore Reach = \sqrt{x^2+y^2}
\end{align}

By using triangle \textit{BCD} and the Cosine rule,

\begin{align}
cos(\alpha) &= \frac{B^2+C^2-D^2}{2\times B\times C}\\
\therefore \alpha &= cos^{-1}\Bigg(\frac{B^2+C^2-D^2}{2\times B\times C}\Bigg).
\end{align}

The value of $E_x$ can be calculated similar to that of reach, but first finding the $x$ and $y$ component lengths of segment A as shown in Figure \ref{fig:Leg_design_3}.

\begin{align}
A_x &= A \times sin(\theta)\\
A_y &= A \times cos(\theta)\\
E_x &= \sqrt{(x-A_x)^2+(y-A_y)^2}
\end{align}
The Sine rule can then be used on triangle $DE_xE_y$ to find,

\begin{align}
\frac{sin(c)}{C} &= \frac{sin(\alpha)}{D}\\
\therefore sin(c) &= C \times \frac{sin(\alpha)}{D}\\
\therefore c &= sin^{-1}\Bigg(C \times \frac{sin(\alpha)}{D}\Bigg).
\end{align}

Angle \textit{e} can easily be found as

\begin{align}
e = tan^{-1}\Bigg(\frac{E_x}{E_y}\Bigg).
\end{align}

With all of this calculated, phi can be calculated as

\begin{align}
\label{eq:IKend}
\phi = 270^o - c -e.
\end{align}

The analysis above is valid for the generic 3 segment design shown in Figures \ref{fig:Leg_design_2} and \ref{fig:Leg_design} where the hip is at the origin of the Cartesian system. The real robot has five legs that need to be calculated for. In order to avoid using five different Cartesian systems, it is necessary to be able to rotate and translate this generic model to suit the requirements of any of the legs.\\

\FloatBarrier
\begin{figure}[h]
\centering
\includegraphics[scale = 1]{pics/Body_Layout_2.pdf}
\caption{Top view of the robot showing individual Cartesian systems for individual legs.}
\label{fig:Body_layout_2}
\end{figure}
\FloatBarrier

Figure \ref{fig:Body_layout_2} illustrates that both leg 1 and 2 have a theta angle of $\theta = 90^o$, yet they are clearly not facing the same direction. This is because of this separate coordinate system that each of the legs have in order to do the inverse kinematic calculations. In order to do calculations on the whole robot, the centre of the robot is used as the origin.\\

If \textit{leg} denotes the number of the desired leg, a leg's coordinate system can be rotated by using

\begin{align}
\label{eq:rotation}
(x',y') &= (x\times cos(\gamma) -y\times sin(\gamma),x \times sin(\gamma) + y \times cos(\gamma))\\
\text{where }\gamma &= (leg-1)\times 72^o. 
\end{align}

For the robot to be able to walk, all five legs should be working together to move in a specific direction. The use of vectors work well since they can be added easily and contain both direction and magnitude information. The movement of each foot can be constrained for each of the bends $(\theta,\phi,\alpha)$ to limit the angle to a realistic value for the servo. These limitations can be seen in Figure \ref{fig:Body_layout_3} as dotted lines. The dashed line represents the vector for movement. The movement vector is limited by the boundary.

\FloatBarrier
\begin{figure}[h]
\centering
\includegraphics[scale = 1]{pics/Body_Layout_3.pdf}
\caption{Top view of the robot showing the addition of vectors and leg boundaries.}
\label{fig:Body_layout_3}
\end{figure}
\FloatBarrier

All of the legs move together towards a point inside or on the edge of each individual perimeter. The direction of movement of the robot can be calculated as

\begin{align}
angle = arctan \Bigg(\frac{y}{x}\Bigg),
\end{align}
where $x$ and $y$ denote the input from the user interface. The magnitude can be found to be
\begin{align}
magnitude = \sqrt{x^2+y^2}.
\end{align}
The magnitude and direction can be used to check if the vector fits inside the perimeter.\\

All of the calculations above only account for translation, that is moving forward, backward, left, right or a combination of the above. Being able to move truly holonomically means being able to rotate with or without translating. The vectors used to calculate the trajectory of each leg works well for representing translation but does not work so well for rotation. Rotation can instead be expressed as a scalar value, which has magnitude and a sign indicating the direction of rotation. A positive value indicates a clockwise rotation. When these rotation vectors are applied to the individual legs, it results in a different vector being added to each leg indicating the trajectory required by each leg to rotate the robot body. Figure \ref{fig:Body_layout_4} shows the effect of adding a rotation scalar to the movement. The dashed arrows indicate the trajectory of each individual leg.

\FloatBarrier
\begin{figure}[h]
\centering
\includegraphics[scale = 1]{pics/Body_Layout_4.pdf}
\caption{Top view of the robot showing the addition of vectors for rotation about the origin.}
\label{fig:Body_layout_4}
\end{figure}
\FloatBarrier

If each leg were instructed to follow the dashed arrows shown in Figure \ref{fig:Body_layout_4}, the legs would fight each other for grip since they vectors will force them to spread out. Since this is not a controlled movement. Simply using the instantaneous trajectory as the movement vector will not work like it did in the case of translational movement. Instead the trajectory will constantly be changing to keep rotate the body while keeping the feet in their positions. This change can be seen in Figure \ref{fig:Body_layout_5}. The dotted lines represent this constantly changing vector. This can be calculated by rotating the foot position around the origin of the robot. It is important to note that although the legs will be spreading out as the robot body rotates, the feet should stay exactly where they are until the foot is out of the perimeter that the system allows. At this point the robot will lift up the leg and place it in the neutral position to allow further rotation.

\FloatBarrier
\begin{figure}[h]
\centering
\includegraphics[scale = 1]{pics/Body_Layout_5.pdf}
\caption{Top view of the robot showing the addition of curved approximation for rotation about the origin.}
\label{fig:Body_layout_5}
\end{figure}
\FloatBarrier

Since a curve can not be added to  the vector of translation, the curve will need to be broken up into small segments of straight pieces approximating the curved line. These straight pieces can then be added to the translation as vectors. Rotating the coordinates around the robot is not a simple task because of the shifting Cartesian systems. In order to rotate about a point that is not the origin of the plane, the coordinate system needs to be translated to move the origin, rotated around this origin and then shifted back to the original origin.

\FloatBarrier
\begin{figure}[h]
\centering
\includegraphics[scale = 1]{pics/Body_Layout_6.pdf}
\caption{Illustration showing the procedure of rotation about the origin.}
\label{fig:Body_layout_6}
\end{figure}
\FloatBarrier

Figure \ref{fig:Body_layout_6} attempts to explain the robot rotation procedure. The following features appear on the sketch:
\begin{itemize}
\item Point \textit{O} is the origin, the centre of the robot.
\item Point \textit{A} is the shoulder of the relevant leg.
\item Point \textit{P} is the position coordinate of the foot before any rotation.
\item Point \textit{P*} is the position coordinate of the foot after rotation.
\item $\beta$ is the rotation angle.
\item \textit{r} is the robot radius.
\item The dashed line is the rotation path of the foot.
\end{itemize}

When the robot needs to rotate, the procedure is therefore as follows.
\begin{enumerate}
\item Start with coordinates of point \textit{P}, relative to point \textit{A} which is the origin of the Cartesian system at this point.
\item Add the coordinates of \textit{P} relative to \textit{A} to the coordinates of \textit{A} relative to \textit{O}. The result is the  coordinates of \textit{P} relative to \textit{O}.
\item Rotate through angle $\beta$ using the rotation formula used in Equation \ref{eq:rotation}.
\item Move the origin back to \textit{A} by subtracting \textit{A} relative to \textit{O}.
\end{enumerate}

The result can be used to perform the Inverse Kinematic calculations as described in Equations \ref{eq:IKstart} through \ref{eq:IKend}

\subsubsection{Simulation}
In order to test the equations and design outlined in the Theory section above, the mathematics was implemented in a Python program consisting of a GUI to enter elementary commands similar to that implemented in the Android application, as well as a window for plotting the result in a 3-Dimensional Cartesian system. Some results showing the validity of the design equations in the previous section can be found below. All of the source code used to build this simulation can be found in Part 5 of this report on the attached optical disc.



\subsubsection{Optimisation}
The design of the algorithm used in the simulation works very well to systematically determine the 15 required servo angles from the three input values $(X,Y,R)$ which ultimately represent the speed for the $X$-direction, $Y$-direction and Rotation. What the simulation fails to take into account is the practical time constraints necessary to make a robot work. In simulation the result was just plotted so all the movement could be done instantaneously. In practice however, the servo motors used to control the limbs of the robot need time to adjust to the set position. These servo motors use traditional analogue proportional, integral, derivative (PID) control techniques. The derivative term means that a sudden, large change in setpoint would result in a large current surge as a result of the PID controller trying to get to the setpoint as soon as possible. If all 15 of the servo motors are updated at the same time with a large jump in position, the surge in current required from each individual servo motor would sum to a large surge, possibly causing a brief dip in the system voltage. The solution to this is to not make large changes in setpoint between updates to the servo motors but rather smaller updates more frequently. The exact time between updates will be a function of a how long it takes the floating point arithmetic (FPU) unit of the microcontroller to do all the floating point math required by the IK algorithm once.

%Electronics
\subsubsection{Electronic design}
In this project, the user interface is implemented in the form of an android application. Commands are communicated to a microcontroller via a serial connection with a Bluetooth module. The microcontroller does all of the calculations and performs the necessary algorithms to determine what should be done with which servo and then communicates this to the relevant servo motor. The algorithm for movement also takes some inputs from the microcontroller into account.\\

The electronics to be implemented for this project is listed below. 
\begin{itemize}
\item Power regulation and supply.
\item Support electronics for the microcontroller.
\item Digital input signal conditioning.
\item Bluetooth module interfacing.
\item Digital output driving where applicable.
\item Servo power control switch
\end{itemize}
Each of these items are discussed separately below.\\

Power regulation and supply for the project can be divided into two parts. The first of these is the low voltage, low power part that supplies the microcontroller, Bluetooth module and support electronics. The second part is the higher voltage, high power part that supplies power exclusively to the servo motors. Both these supplies will be powered by Li-Ion 18650 cells because of their high power density. The nominal voltage of these cells are $3.7V$. The low voltage, low power supply can easily be powered from a single cell making use of a low dropout (LDO) linear regulator to provide the $3.3V$ rail required. A linear voltage regulator is not suited for the high power application, mainly for two reasons. The first of the two is that these regulators are rarely rated for use above $1.5A$. The second is that a linear regulator dissipates power in itself as heat in order to regulate voltage. This means that the voltage drop formed over the device to bring the output voltage down is dissipated in the device. The power dissipated can be quantified by

\begin{align}
P &= V\times I\\
&= (V_{in}-V_{out})\times I
\end{align}

This is fine for low current applications but sufficient cooling quickly becomes a problem at currents in the Ampere range.
A better solution for regulation at high power is making use of a step-down DC-DC converter. The efficiency of this topology is much greater than for linear regulators. Most are in excess of $90\%$ if operated within the design limits. Since DC-DC converters can become very complex and it is far outside the scope of this project, a complete off-the-shelf module was implemented instead of designing and building one from first principles. Figure \ref{fig:PowerSupply} shows the power supply layout for the robot.

\begin{figure}[H]
\centering
\includegraphics[scale = 1]{pics/PowerSupply.pdf}
\caption{Diagram showing the power supply layout for the robot.}
\label{fig:PowerSupply}
\end{figure}

The support electronics for the microcontroller includes everything necessary for it to be able to start after power up and function normally. The application note provided by the manufacturer has detailed instructions and schematics on this and it was implemented as recommended for this project. This includes ceramic capacitors on all of the power supply pins, electrolytic capacitors on the power rails, a high frequency crystal oscillator, timing capacitors, a reset switch, various pull-up and pull-down resistors and a selector for pulling the BOOT pin high or low. More details on this can be fond in the technical documentation section of the report.\\

In order to protect digital input pins on the microcontroller, as well as debouncing input signals from switches, a small interface circuit is used between an input and a digital pin. This is implemented from prior knowledge gained in earlier modules. Figure \ref{fig:InputProtection} shows the schematic for this.

\begin{figure}[H]
\centering
\includegraphics[scale = 1]{pics/InputProtection.pdf}
\caption{Diagram showing the digital input protection circuit for the robot.}
\label{fig:InputProtection}
\end{figure}

The Bluetooth module used in this project does all of the Bluetooth protocol and decryption automatically and makes the data available through the serial interface. This means that the module is powered by the rails of the microcontroller and all that is further required to receive information is to connect the $TX$ pin of the module to the \textit{USART\_RX} pin of the microcontroller.\\

In order to make the actions and decisions of the robot more apparent to the user, a red, green and blue (RGB) LED array is connected to the microcontroller. Different colours can then be used to indicate different states of the robot. Since the LEDs can't be powered directly from a digital output pin, a driver will have to be built to protect both the microcontroller and the LEDs. Figure \ref{fig:LEDDriver} shows the schematic for this. Each LED requires a bipolar junction transistor (BJT) of type PNP. The PNP is required because the RGB LED has a common ground and therefore has to be switched on the positive side. The microcontroller is used to pull the base of the PNP  high or low. The LED will then switch off or on respectively.

\begin{figure}[H]
\centering
\includegraphics[scale = 1]{pics/LEDDriver.pdf}
\caption{Diagram showing the driver circuit for the RGB LED array.}
\label{fig:LEDDriver}
\end{figure}

Servo motors constantly use power to hold the position they are in. When the robot is idle, waiting for a Bluetooth connection, it is unnecessary for the servo motors to be consuming current. The circuit illustrated in Figure \ref{fig:PowerSwitch}  is designed to switch power to a rail dedicated to servo motor supply. It uses a P-channel MOSFET as well as a NPN BJT. The P-channel MOSFET is required to switch the positive rail with a low internal power dissipation due to a low drain-source resistance when switched on. The BJT is necessary to switch it on because the microcontroller pin on its own can't reach the 6V required to completely switch off. When the pin is low, the BJT is switched off, the MOSFET gate is pulled high by the resistor and therefore the MOSFET is off. When the pin is high, the BJT pulls the MOSFET gate low and the MOSFET is fully on, thereby turning the servo rail on.

\begin{figure}[H]
\centering
\includegraphics[scale = 1]{pics/PowerSwitch.pdf}
\caption{Diagram showing the driver circuit for the RGB LED array.}
\label{fig:PowerSwitch}
\end{figure}

\subsubsection{Electronic implementation}

The microcontroller chosen is an STM32F576GTV6. This specific model was chosen because of its powerful FPU, high clock speed and the large pin count for possible expanding later. This unit is, however, only available in an LQFP100 package with 0.4mm pin spacing. This was soldered to a breakout board to make interfacing with the hardware easier. The breakout board was glued to a Veroboard where the rest of the electronics, discussed above, was implemented.

\captionsetup[table]{position=bottom}
%Software
\subsubsection{Software design}
\label{sec:soft}
This section contains a summary of the development process for the software algorithm implemented in both the simulations discussed in section \ref{sec:sim} and the final embedded software implementation.\\

If all of the legs are moved and lifted together the feet would stay in exactly the same position while the body moves in a circular pattern. The key to taking steps and therefore walking is lifting only one leg and moving it while the other legs remain in position. The trajectories calculated in section \ref{sec:theory} will be used and broken into smaller steps which are executed in fixed intervals to maintain a specific speed. 

Figure \ref{fig:Soft1} attempts to explain the rough algorithm used to reset the legs and therefore the algorithm required for walking. This is the algorithm implemented in the Python program for simulation without the plotting functions.

\begin{figure}[H]
\centering
\includegraphics[scale = 1]{pics/Soft1.pdf}
\caption{Flow diagram showing robot walking algorithm.}
\label{fig:Soft1}
\end{figure}

While this is the core of the algorithm, some additional functions need to be included to make provision for some conditions. These are:
\begin{itemize}
\item Power is turned on initially.
\item Power is lost while in operation.
\item Power is regained.
\item Bluetooth is initially connected.
\item Bluetooth is suddenly disconnected.
\end{itemize}

A more complete diagram illustrating the design implemented on the microcontroller can be seen in Figure \ref{fig:Soft2}.\\

Servos are controlled with a square pulse with a specific width. The standard for hobbyist servo motors with a $180^o$ movement range is that a high pulse of $1ms$ equates to $0^o$ while $2ms$ results in $180^o$. The period of this signal is $20ms$.\\

A typical set of control signals for an analogue servo motor can be seen in figure \ref{fig:Servo1}.

\begin{figure}[H]
\centering
\includegraphics[scale = 1]{pics/Servo1.pdf}
\caption{Illustration of the working of a traditional servo control signal.}
\label{fig:Servo1}
\end{figure}

\begin{figure}[H]
\centering
\includegraphics[scale = 1]{pics/Soft2.pdf}
\caption{Flow diagram showing robot walking algorithm final implementation.}
\label{fig:Soft2}
\end{figure}

In order to control the 15 servo motors on the robot simultaneously, 15 independent control signals like the ones shown in Figure \ref{fig:Servo1}. A common way to do this is to use the microcontroller's built-in pulse width modulation (PWM) module with a fixed frequency and variable duty cycle. While this is a valid approach for most cases, most microcontrollers don't have nearly enough independent PWM channels available to facilitate the 15 servo motors in this application.\\

Off-the-shelf modules that have a large number of PWM channels available and communicate with the microcontroller using the I2C, ISP or serial protocols are available but increases the electronic hardware components and power consumption.\\

The preferred solution in this case is using 15 normal digital output pins and two internal timers. One of the timers is set up to interrupt at $50Hz$, the period of the control signals. The second timer is reconfigured after each interrupt to have a new period. Figure \ref{fig:Servo2} attempts to illustrate how this procedure works for the simpler case of 5 servos.\\

\begin{table}[H]
\centering
\caption{Servo angles for the example explaining servo control using two timers.}
\label{tab:servo}
\begin{tabular}{ccc}
\textbf{Servo} & \multicolumn{1}{l}{\textbf{Angle (deg)}} & \multicolumn{1}{l}{\textbf{Timer (ms)}} \\ \hline
1              & 45                                       & 1.25                                \\
2              & 90                                      & 1.5                                    \\
3              & 180                                      & 2                                       \\
4              & 0                                        & 1                                       \\
5              & 90                                      & 1.5                                    
\end{tabular}
\end{table}

Before starting the procedure, Table \ref{tab:servo} is sorted in ascending order by the period in ms. The result can be seen in Table \ref{tab:servo2}

\begin{table}[H]
\centering
\caption{Sorted servo angles for the example explaining servo control using two timers.}
\label{tab:servo2}
\begin{tabular}{cccc}
\textbf{Index} & \textbf{Servo} & \textbf{Angle (deg)} & \textbf{Timer (ms)} \\ \hline
1              & 4              & 0                    & 1                   \\
2              & 1              & 45                   & 1.25                \\
3              & 2              & 90                   & 1.5                 \\
4              & 5              & 90                   & 1.5                 \\
5              & 3              & 180                  & 2                  
\end{tabular}
\end{table}

For the simplified case where 5 servo motors should be controlled, Table \ref{tab:servo} shows example values to illustrate the procedure. The procedure is outlined using the event letters shown in Figure \ref{fig:Servo2}.
\begin{enumerate}[label = \Alph*]
\item Timer 1 interrupts. All servo pins are set high. Timer 2 period is set to the timer value for index 1 (a=$1.25ms$).
\item Timer 2 interrupts. Servo for index 1 is set low (Servo 4). Timer 2 period is set to the timer value in index 2 - timer of index 1 (b=$1.25-1 = 0.25ms$).
\item Timer 2 interrupts. Servo for index 2 is set low (Servo 1). Timer 2 period is set to the timer value in index 3 - timer of index 2 (c=$1.5-1.25 = 0.25ms$).
\item Timer 2 interrupts. Servo for index 3 and 4 is set low (Servo 2 and 5). Timer 2 period is set to the timer value in index 5 - timer of index 4 (d=$2-1.5 = 0.5ms$).
\item Timer 2 interrupts. Servo for index 5 is set low (Servo 3). Timer 2 is turned off for the remainder of the $20ms$ ($e=NULL$)
\item Timer 1 interrupts. All servo pins are set high. Timer 2 period is set to the timer value for index 1 (a=$1.25ms$).
\item Timer 2 interrupts. Servo for index 1 is set low (Servo 4). Timer 2 period is set to the timer value in index 2 - timer of index 1 (b=$1.25-1 = 0.25ms$).
\item Timer 2 interrupts. Servo for index 2 is set low (Servo 1). Timer 2 period is set to the timer value in index 3 - timer of index 2 (c=$1.5-1.25 = 0.25ms$).
\item Timer 2 interrupts. Servo for index 3 and 4 is set low (Servo 2 and 5). Timer 2 period is set to the timer value in index 5 - timer of index 4 (d=$2-1.5 = 0.5ms$).
\item Timer 2 interrupts. Servo for index 5 is set low (Servo 3). Timer 2 is turned off for the remainder of the $20ms$ ($e=NULL$)
\end{enumerate}

\begin{figure}[H]
\centering
\includegraphics[scale = 1]{pics/Servo2.pdf}
\caption{Illustration of how multiple servo motors are controlled using two timers.}
\label{fig:Servo2}
\end{figure}

\subsubsection{Software implementation}
%Hardware
\subsubsection{Hardware design}

\subsubsection{Hardware implementation}

\subsubsection{Design summary}
\begin{table}[H]
\centering

\begin{tabular}{|p{5cm}|p{5cm}|p{5cm}|}
\hline
\textbf{Task} & \textbf{Implementation} & \textbf{Task complete} \\
\hline
 Development of an IK engine in software & The mathematical analysis of the robot was used to create a generic model, which was then first implemented in Python and thereafter in the final robot implementation. & Completed \\\hline
 
Development of a walking algorithm  & A walking algorithm was developed and optimized in Python and then implemented in the final robot.  & Completed\\\hline

Development of a user interface & The user interface was developed for an Android smartphone using Android Studio. & Completed \\\hline

Communication channel between user interface and robot & Bluetooth was used to establish a low bandwidth, low cost communication system between the smartphone and the robot. It communicates commands successfully. & Completed \\\hline

Implementation of electronics on a PCB.& All electronics were implemented on a Veroboard instead of a PCB which is inexpensive and sufficiently robust for the purpose of this project. A PCB would not have made any functional difference to the end result and is not suited for making changes while prototyping. The only difference would have been aesthetically.& Incomplete \\\hline

Implementation of inclinometer sensor & An inclinometer was planned to give the robot the ability to adapt to the exact angle of slanted surfaces but this was not implemented because the robot walks without a problem on the angles of surfaces specified for this project. & Incomplete \\\hline
\end{tabular}
\caption{Design summary}
\label{my-label}
\end{table}


\secun{Results}

\subsubsection{Summary of results achieved}
\begin{table}[H]
\begin{tabular}{|p{5cm}|p{5cm}|p{5cm}|}
\hline
\textbf{Description of requirement or specification
(intended outcome)} & \textbf{Actual outcome}& \textbf{Location in report}\\
\hline
\multicolumn{3}{|l|}{}\\
\multicolumn{3}{|l|}{\textbf{Mission requirements of the product}}\\
\hline

The robot should be able to move in a given direction in 30 degree increments from a stationary position. & The robot is able to move holonomically, therefore able to move in 30 degree increments.&4.2.1\\\hline

The robot should be able to rotate 90 degrees without translating the centre of the body by more than 5\% of the body diameter.&The robot is able to do this.&4.2.2\\\hline

The time it requires to complete a manoeuvre should not vary more than 25\% between smooth and rough surfaces.&The time differs by about 20\%&4.2.3\\\hline

The robot should be able to walk at a speed of at least 100mm/s.&The robot is only capable of 8mm/s&4.2.4\\\hline

The robot should be able to walk on a surface with a 10\% incline without falling over.&The robot can successfully walk on a slanted surface with an incline of up to 10\% &4.2.5\\\hline

\multicolumn{3}{|l|}{}\\
\multicolumn{3}{|l|}{\textbf{Field conditions}}\\
\hline
The robot should be in range of the wireless smartphone controller.&The robot works fine if it is within a range of 10m of the smartphone&\\\hline

To prevent falling over, the robot should walk on surfaces close to horizontal.&The robot is able to work on slanted surfaces with an incline of less than 10\%&\\\hline

The robot should stay dry to protect electronics.&The robot was never tested in wet conditions&\\\hline

The robot should work in normal temperature conditions for South Africa.&The robot never experienced any issues during the testing phase that relates to ambient temperature, relative humidity or any other atmospheric conditions&\\\hline

\end{tabular}
\caption{Summary of results achieved}
\label{tab:sum}
\end{table}

\subsubsection{Summary of results achieved}
\begin{table}[H]
\begin{tabular}{|p{5cm}|p{5cm}|p{5cm}|}
\hline

\multicolumn{3}{|l|}{}\\
\multicolumn{3}{|l|}{\textbf{Specifications}}\\
\hline
The smartphone application should communicate commands from the user interface to the robot at a frequency of at least 10 Hz.&The smartphone application does do this&\\\hline

The inverse kinematics calculator of the robot should be able to calculate the joint positions correctly to move each leg to the desired location.& This works as designed&\\\hline

The servo motor angles should all be within 5 degrees from the calculated values.& The servo motors are all accurate to within 5 degrees&\\\hline

\multicolumn{3}{|l|}{}\\
\multicolumn{3}{|l|}{\textbf{Deliverables}}\\
\hline
Control code for inverse kinematics and servo control.&This was completed and is functional&\\\hline

Android application with a user interface for control of the robot.&This was completed and is functional&\\\hline

Circuits implemented on PCB for interfacing all hardware with the microcontroller.&This was performed on Veroboard instead of PCB&\\\hline

Robot body and legs.& This was all successfully 3D printed&\\\hline

Simulations on all implemented software and analogue design& Tis was completed and is functional&\\\hline

\end{tabular}
\caption{Summary of results achieved (continued)}
\label{tab:sum2}
\end{table}

\subsubsection{Qualification tests}

%Test 1
\paragraph{Qualification test \arabic{paragraph} : Holonomic translation}
\label{par:1}
\subparagraph{Qualification test}
\textit{Objectives of test/experiment}\\
The objective is to prove that the robot is capable of moving in any direction from a standing position without first rotating.
\textit{Equipment used}\\
A canvas with lines in 30 degree increments is used to put the robot on.
\textit{Experimental parameters and setup }\\
The robot is turned on and placed in the centre of the canvas. The test is conducted in the field conditions listed in Table \ref{tab:sum}.
\textit{Experimental protocol}\\
The robot is instructed to move along one of the lines and then return to the centre by using the smartphone as remote control. Once back in the centre, the next line is followed. This process continues until all the lines have been followed.
\subparagraph{Results and observations}
\textit{Measurements}\\
Figure \ref{fig:Res1} shows the robot on the canvas where the qualification test is being performed.
\begin{figure}[H]
\centering
\includegraphics[scale = 1]{pics/Res1.jpg}
\caption{Top view of the robot on the canvas.}
\label{fig:Res1}
\end{figure}
\textit{Description of results}\\
The robot is able to successfully follow the lines on the canvas without ever rotating.

%Test 2
\paragraph{Qualification test \arabic{paragraph} : Rotation without translation}
\subparagraph{Qualification test}
\textit{Objectives of test/experiment}\\
The objective of this test is to confirm that the robot is capable of rotating around its own axis without translating.
\textit{Equipment used}\\
The same canvas used in \ref{par:1} is used for this qualification test. It has two circles, one has the diameter of the robot base, the second is 5\% larger.
\begin{figure}[H]
\centering
\includegraphics[scale = 1]{pics/Res2.jpg}
\caption{Canvas used for this experiment.}
\label{fig:Res2}
\end{figure}
\textit{Experimental parameters and setup }\\
The robot is placed inside the smaller circle and switched on. The test is conducted in the field conditions listed in Table \ref{tab:sum}.
\textit{Experimental protocol}\\
The robot is instructed to rotate from the smartphone application. After 90 degrees of rotation, the robot is instructed to stop and is then turned off. The robot should be within the bigger circle.
\subparagraph{Results and observations}
\textit{Measurements}\\
\begin{figure}[H]
\centering
\includegraphics[scale = 1]{pics/Res3.jpg}
\caption{Robot inside the larger circle}
\label{fig:Res3}
\end{figure}
\textit{Description of results}\\
The robot is still inside the larger circle after rotating 90 degrees.

%Test 3
\paragraph{Qualification test \arabic{paragraph} :Speed on rough terrain}
\subparagraph{Qualification test}

\textit{Objectives of test/experiment}\\The purpose of this test is to show that the robot functions reasonably well on rough terrain when compared to a smooth surface.
\textit{Equipment used}\\
A small test track developed for this purpose is used. In order to compare performance between the smooth and rough terrain, the time to complete the track is recorded with a stopwatch application on a smartphone.
\textit{Experimental parameters and setup }\\
A rough terrain is simulated by scattering small, loose obstacles (foam blocks) over the test track. This is a realistic representation of a wide variety of terrain conditions. For preparation the robot is placed on the start line of the smooth test track and switched on. The test is conducted in the field conditions listed in Table \ref{tab:sum}.
\textit{Experimental protocol}\\
The stopwatch is started and the robot is instructed to follow the line that forms the test track. When the robot reaches the end, the time on the stopwatch is recorded. The test track is scattered with the small obstacles and the procedure is repeated.
\begin{figure}[H]
    \centering
    \begin{subfigure}[t]{0.5\textwidth}
        \centering
        \includegraphics[scale = 1]{pics/Res5.jpg}
        \caption{Test track without obstacles.}
    \end{subfigure}%
    ~ 
    \begin{subfigure}[t]{0.5\textwidth}
        \centering
        \includegraphics[scale = 1]{pics/Res4.jpg}
        \caption{Test track with obstacles.}
    \end{subfigure}
    \caption{Test setup for proving that the robot can easily walk on rough surfaces}
    \label{fig:Res4}
\end{figure}
Figure \ref{fig:Res4} shows the test track for the two different tests.
\subparagraph{Results and observations}
\textit{Measurements}\\
\begin{table}[H]
\centering
\begin{tabular}{|l|l|}
\hline
\textbf{Course condition}&\textbf{Time}\\\hline
Smooth&5:12\\\hline
Rough & 6:20\\
\hline
\end{tabular}
\caption{Results achieved for the test track}
\label{my-label}
\end{table}
\textit{Description of results}\\
The robot moves slower over the obstacles than on the smooth course.

%Test 4
\paragraph{Qualification test \arabic{paragraph} : Speed}
\subparagraph{Qualification test}
\textit{Objectives of test/experiment}\\
The objective of this test is to prove that the robot can reach a speed of 100mm/s
\textit{Equipment used}\\
A linear track designed for this experiment is used to measure the speed. It has markings that is a fixed distance apart that makes it easy to measure the distance travelled. The time will be recorded using a stopwatch application on a smartphone. 
\begin{figure}[H]
\centering
\includegraphics[scale = 1]{pics/Res6.jpg}
\caption{Robot on the track for measuring speed.}
\label{fig:Res6}
\end{figure}
Figure \ref{fig:Res6} shows the robot on the track mentioned above.
\textit{Experimental parameters and setup }\\
The robot is placed at the start of the track and switched on. The test is conducted in the field conditions listed in Table \ref{tab:sum}.
\textit{Experimental protocol}\\
The stopwatch is started and the robot is instructed to move down the track as fast as possible. When the end of the track is reached, the time on the stopwatch is recorded.
\subparagraph{Results and observations}
\textit{Measurements}\\
\begin{table}[H]
\centering
\begin{tabular}{|l|l|l|}
\hline
\textbf{Distance}&\textbf{Time}&\textbf{Speed}\\\hline
1.5m&3:07&8mm/s\\\hline
\end{tabular}
\caption{Results achieved for the speed test track}
\label{my-label}
\end{table}
\textit{Description of results}\\
The robot moves much slower than anticipated.

%Test 5
\paragraph{Qualification test \arabic{paragraph} :Slopes}
\subparagraph{Qualification test}
\textit{Objectives of test/experiment}\\
The objective is to prove that the robot is capable of walking on a slanted surface without falling over.
\textit{Equipment used}\\
A plank is used to create the slanted surface. A long ruler is used to measure the incline of the plank.
\textit{Experimental parameters and setup }\\
The plank is angled at 10\%. This is ensured by making sure that the plank has a height of 10cm at the point that is 1m from the edge of the plank on the ground. The robot is put at the base of the plank. The test is conducted in the field conditions listed in Table \ref{tab:sum}.
\textit{Experimental protocol}\\
The robot is instructed to walk on the plank to see if it struggles or falls over at any stage in the process.
\subparagraph{Results and observations}
\textit{Description of results}\\
The robot walks without any additional difficulty due to the incline. The robot was never close to falling over at any stage in the experiment.

%Test 6
\paragraph{Qualification test \arabic{paragraph} :Slippery surfaces}
\subparagraph{Qualification test}
\textit{Objectives of test/experiment}\\
The objective is investigate the performance of the robot on a slippery surface
\textit{Equipment used}\\
A wooden floor is used as the slippery surface for the robot to walk on.
\textit{Experimental parameters and setup }\\
The robot is put on the wooden floor and turned on. The test is conducted in the field conditions listed in Table \ref{tab:sum}.
\textit{Experimental protocol}\\
The robot is instructed to walk on the wooden floor in both slow and rapid movements to investigate how the legged design functions on low traction environments.
\subparagraph{Results and observations}
\textit{Description of results}\\
The robot with more difficulty on slippery surfaces than on surfaces withh higher traction.

\secun{Discussion}
\input{Disc.tex}

\secun{Conclusion}
\subsubsection{Summary of the work}
The mathematical analysis of the generic robot body was done in such a way that the model could be changed parametrically to allow for greater flexibility in the design process. This was applied and developed in a Python simulation which eventually ended up to include a user interface and a plot of the robot body in action. Once the walking algorithm was implemented and optimized, the work was implemented in C to be used in the microcontroller that would control the robot. Algorithms were designed and implemented to control all 15 of the servo motors with as little hardware and software resources required as possible. An android application was developed to be a user interface and act as a remote control for the robot. Instructions are sent to the robot using Bluetooth, which are received by a Bluetooth module connected to the microcontroller with a serial connection. Legs and a body was 3D-printed to fit the servo motors and the robot was able to move its first legs. The movement of legged vehicles on slippery surfaces was investigated and the results were surprising. The investigation was somewhat limited by the available torque from the servo motors.\\
\subsubsection{Summary of the observations and findings}
The robot is able to walk but not in the way originally intended. Although holonomic motion is possible and those design goals have been met, the robot struggles to move properly. This is a result of the robot being much heavier than anticipated. The extra strain on the servo motors causes a high power demand from the servo motors which the DC-DC converter is not always able to supply. The result in the rail voltage is that dips occur when the robot requires strength for more than one leg at a time. The dip in voltage reduces the servo torque, making the situation even worse. A partial solution to this is to keep the robot as light as possible and severely limit the speed of all movements. This somewhat helps but the robot is no longer able to meet the mission critical specification for speed. The movement accuracy also suffers from the same problem.\\
\subsubsection{Suggestions for future work}
The work could be continued and largely improved by getting the robot to walk properly without exchanging the servo motors with stronger ones. This could possibly be done by changing the gear ratio on the outer two limbs to sacrifice some range for torque. This will allow the design to keep the low power consumption it was aiming for while being able to work as intended in the design.

