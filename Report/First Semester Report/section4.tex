%%
%%  Department of Electrical, Electronic and Computer Engineering.
%%  EPR400/2 Project Proposal - Section 4.
%%  Copyright (C) 2011-2017 University of Pretoria.
%%

\section{Progress}

In semester 1, the mathematical analysis of the robot structure was completed and this was used to develop the formulae that is used to perform the inverse kinematics required during normal operation. The inverse kinematic equations was developed from first principles using the mathematical structure analysis and trigonometry.\\

The parts required to build the robot was designed in a CAD package to be as strong and light as possible without compromising on functionality. It is important for these parts to be light as this results in less torque required by the stepper motors. It is also important because the parts are 3D printed and lighter parts result in lower cost. The parts were all sanded and filed after printing to make the parts smoother and make the fit between parts better with less friction between joints.\\

In order to test the inverse kinematic equations and preliminary algorithms, a GUI was developed in Python. The program takes inputs from a cluster of buttons similar to that of a remote control. The programs plot a diagram in 3D that represents the actual robot. This will be used to tune the final walking algorithm as well.\\

An Android application was developed to communicate instructions to the robot via Bluetooth. The user interface consists of two on-screen joysticks as well as a few buttons that is used to control the movements of the robot.\\

\newpage

%% End of File.


