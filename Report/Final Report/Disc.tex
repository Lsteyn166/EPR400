\newpage
\subsubsection{Interpretation of results}
The results recorded in the previous section of this report indicate that the robot is able to function as a holonomic robot, therefore it is able to move in any direction without first rotating and rotate without translating. The results that measure the performance of the robot, however, show that it does not do it nearly as good as it was designed to do it. The maximum speed of the robot, for example, is far below what was aimed for originally. The fact that the robot is able to correctly execute commands shows that the theoretical design and algorithm is valid. In the case of the speed requirement, it is the mechanical design that was lacking. The robot was far heavier that planned for and the servo motors used struggle to carry the load. The only way to make the robot function is to slow all of the movements down. The test on slippery surfaces had unexpected results as well. The expectation was that the robot might struggle to walk because the legs would slip on the floor and it would slip when propelling itself forward. Instead the low friction causes the legs to want to spread outwards, causing more strain on the servo motors. The choice of servo motor implemented in the robot caused the robot to not work as good as it should have. If more powerful servo motors were implemented, the power consumption would have been significantly higher but torque would have been sufficient.  

\subsubsection{Aspects to be improved}
The mechanical drive system for the legs could be improved. Torque at the critical joint is not sufficient and the full range of motion is not used. A solution that would not require a complete redesign would be to change the gear ratio at the critical joint to have a more powerful actuator with a smaller range of motion. The alternative solution would be to design the robot with larger and more powerful servo motors in the first place. This would, however, require a redesign of the power management system for the servo motors as well since larger servos require more current.

\subsubsection{Strong points}

The strongest part of this project is the inverse kinematics and motion planning algorithm. This is the case because it was designed for a parametric model which made it extremely flexible in the design phase. It was also thoroughly tested and optimized in the simulation UI in Python well before it was first implemented in the robot.

\subsubsection{Under which circumstances will the current system fail?}
The largest risk of failure of this robot is the strained servo motors. If any weight is added to the system, it becomes more difficult to move. This also places restrictions on the speed of the movements. The strained servo motors require a current closer to the stall current than the system was designed for. This means that the supply rail of the servo motors dip in voltage if the speed of motions is too high, causing the system to fail. The system would fail safe rather than violently. If a servo stalls, the batteries are protected internally against too high discharge current as well as over discharge and would simply shut down in either of these cases.

\subsubsection{Design ergonomics}
The user interface is designed to be as simple to use as possible and therefore only has the two joystick-like controls on the screen. Both can be used at the same time. The advantage of joysticks over a directional pad is that the user does not have to look at the display when using it.
 
\subsubsection{Health and safety aspects of the design}
The Lithium-Ion batteries chosen for implementation in this project have a built-in protection against over-charging, over-discharging, over-current and over-temperature. This makes the use of Lithium-Ion batteries much safer.
\subsubsection{Social and legal impact and benefits of the design}
There is no specific legislation that the product has to comply with. If the robot is able to we controlled remotely using longer range technologies, the methods used in this project could be used to get help or supplies to people in need in dangerous situations. This could be getting food or supplies to people in war zones or navigating tough terrain during search-and-rescue operations.

\subsubsection{Environmental impact and benefits of the design}
The product is designed to have as little impact as possible on the environment during use and when the end of its lifetime is reached. All of the plastic used for creating the mechanical parts are created using PLA plastic, which is bio-degradable. The Lithium-Ion batteries can be handed in for recycling at any drop-off point for used batteries found in most shopping malls. The stainless steel parts such as bearings and rods can be re-used in another project. The only part that poses a risk is the electronic waste. This includes the Veroboard, PCBs of modules and the electronic components. These should be dropped off somewhere where the recycling is specialized for the handling of electronic waste.